%  sample eprint article in LaTeX           --- M. Peskin, 9/7/00
\documentclass[12pt]{article}
\usepackage{graphicx}

%%%%%%%%%%%%%%%%%%%%%%%%%%%%%%%%%%%%%%%%%%%%%%%%%%%%%%%%%%%%%%%%%%%%
% basic data for the eprint:
%%%%%%%%%%%%%%%%%%%%%%%%%%%%%%%%%%%%%%%%%%%%%%%%%%%%%%%%%%%%%%%%%%%%

\textwidth=6.0in  \textheight=8.25in

%%  Adjust these for your printer:
\leftmargin=-0.3in   \topmargin=-0.20in

%% preprint number data:
\newcommand\pubnumber{DRAFT-001}
\newcommand\pubdate{\today}

%%  address and funding acknowledgement data:
\def\oxford{}
\def\support{\footnote{}}

%%%%%%%%%%%%%%%%%%%%%%%%%%%%%%%%%%%%%%%%%%%%%%%%%%%%%%%%%%%%%%%%%%%%%%%%%%%%
%   document style macros
%%%%%%%%%%%%%%%%%%%%%%%%%%%%%%%%%%%%%%%%%%%%%%%%%%%%%%%%%%%%%%%%%%%%%%%%%%%%
\def\Title#1{\begin{center} {\Large #1 } \end{center}}
\def\Author#1{\begin{center}{ \sc #1} \end{center}}
\def\Address#1{\begin{center}{ \it #1} \end{center}}
\def\andauth{\begin{center}{and} \end{center}}
\def\submit#1{\begin{center}Submitted to {\sl #1} \end{center}}
\newcommand\pubblock{\rightline{\begin{tabular}{l} \pubnumber\\
         \pubdate  \end{tabular}}}
\newenvironment{Abstract}{\begin{quotation}  }{\end{quotation}}
\def\Acknowledgements{\bigskip  \bigskip \begin{center} \begin{large}
             \bf ACKNOWLEDGEMENTS \end{large}\end{center}}
%%%%%%%%%%%%%%%%%%%%%%%%%%%%%%%%%%%%%%%%%%%%%%%%%%%%%%%%%%%%%%%%%%%%%%%%%%%%
%  personal abbreviations and macros
%    the following package contains macros which can be used in this document:
%\input econfmacros.tex
%%%%%%%%%%%%%%%%%%%%%%%%%%%%%%%%%%%%%%%%%%%%%%%%%%%%%%%%%%%%%%%%%%%%%%%%%%%

\begin{document}
\begin{titlepage}
\hspace{-1.4cm}\pubblock

\vfill
\Title{Feasibility Study for the Final State $hh\rightarrow (b\bar{b})(b\bar{b})$}
\vfill
\Author{}
\Address{}
\vfill
\begin{Abstract}
This is a feasibility study for the final state $hh\rightarrow (b\bar{b})(b\bar{b})$ using reconstruction techniques for both resolved
and boosted topologies as well as multivariate methods.
\end{Abstract}
\vfill
\end{titlepage}
\def\thefootnote{\fnsymbol{footnote}}
\setcounter{footnote}{0}
%

\section{Introduction}


\section{Samples}
\subsection{Background}
All background samples are generated with the {\tt SHERPA} event generator, version 2.1.1. For the explicit runcards used in the generation, see Appendix~\ref{app:runcards}.

The NNPDF 3.0 $n_f = 4$ LO set with strong coupling $\alpha_S=0.118$ is used for all samples. At the generator level the following basic cuts are applied. Each final state particle must have $p_T \ge 20$ GeV, and be located within $| \eta | \le 3.0$. Factorisation and renormalisation scales are set as $\mu_F=\mu_R=H_T/2$ (in hindsight, $H_T/4$ would probably have made more sense).

\section{Object and Event Selection}

\subsection{Resolved Topology}

The resolved selection requires the presence of at least four b-tagged anti-$k_T$ $R=0.4$ jets with $p_T >$25~GeV and $|\eta|>2.5$.
\textit{Pre-cut} histograms are written out for various kinematic distributions for all events with at least four b-tagged jets but before applying
further kinematic cuts. \textit{Post-cut} histograms contain only events where the four jets also pass the above kinematic cuts.

The di-Higgs system is reconstructed by considering all possibilities of forming two pairs of jets with invariant masses $m_{j1j2}$ and 
$m_{j3j4}$, respectively, and chosing the configuration that minimises their difference $|m_{j1j2} - m_{j3j4}|$. Only the four leading-$p_T$ jets
are considered here.

\subsection{Boosted Topology}\label{sec:Boosted_FR}

In the boosted topology, the decay products of each Higgs boson are merged into a single large-$R$ jet with a two-prong substructure. 
These Higgs jets are reconstructed as anti-$k_T$ $R=1.0$ jets for which two substructure variables, the first $k_T$ splitting scale $d_{12}$
and the 2-subjettiness ratio $\tau_{21}$, are calculated.
\textit{Pre-cut} histograms are written out after requiring the presence of at least two large-$R$ in the event but without applying any further
kinematic cuts. \textit{Post-cut} histograms are filled after the following additional cuts: 
Both jets are required to have $p_T >$100~GeV and $|\eta|>2.5$. Moreover, each jet is required to have at least two b-tagged anti-$k_T$ $R=0.3$ 
jets matched to it via \textit{ghost association}: 
To this end, the constituents of a given large-$R$ jet is reclustered using the jet algorithm 
and radius parameter of the original jet and so-called \textit{ghost jets}, each corresponding to a small-$R$ jet in the event, 
are added to the input for the cluster sequence. The 4-vector of a ghost jet is obtained from a given small-$R$ jet 
by setting its transverse momentum and mass to negligibly small values but retaining its direction in $\eta$ and $\phi$. A small-$R$ jet is 
considered matched to the large-$R$ jet if its ghost is found among the constituents of the reclustered jet. Ghost association allows to unambiguously
match small-$R$ to large-$R$ jets, even in dense environments where a simple matching based on the distance $\Delta R$ between jets 
may lead to multiple matchings.

\subsection{Boosted Topology with Variable-$R$ Jets}

The same selection as in Subsection~\ref{sec:Boosted_FR} is applied but this time using Variable-$R$ jets with the following parameters:
$\rho=500$~GeV, $R_{max}=$1.0, $R_{min}=$0.2.

\section{Multivariate Tools}

\section{Results}

\section{Conclusion}


% \Acknowledgements


%%%%%%%%%%%%%%%%%%%%%%%%%%%%%%%%%%
%%%%%%%%%%%%%%%%%%%%%%%%%%%%%%%%%%
\clearpage
\appendix
\section{Event sample runcards}
\label{app:runcards}
\subsection {Background: QCD 4b}
\begin{verbatim}
(run){
  EVENTS 3M;
  EVENT_GENERATION_MODE U;
  ME_SIGNAL_GENERATOR Comix;

  EVENT_OUTPUT HepMC_Short[SHERPA_QCD_4b];

  BEAM_1 2212; BEAM_ENERGY_1 7000;
  BEAM_2 2212; BEAM_ENERGY_2 7000;

  FRAGMENTATION=Off # disable hadronisation
  MI_HANDLER=None # disable multiple parton interactions

  SCF:=1; ### default scale factor
  SCALES VAR{SCF*H_T2/2};

  PDF_LIBRARY LHAPDFSherpa;
  PDF_SET NNPDF30_lo_as_0118_nf_4.LHgrid;
  PDF_SET_VERSION 0;

  MASSIVE[5] 1;
  MASS[5] 4.75;

}(run);
(processes){
  Process 93 93 -> 5 -5 5 -5;
  Order_EW 0;
  End process;

}(processes);
(selector){
  PT 5 20 7000
  PT -5 20 7000
  PseudoRapidity 5 -3.0 3.0
  PseudoRapidity -5 -3.0 3.0
}(selector);
\end{verbatim}

\subsection {Background: QCD 2b2j}
\begin{verbatim}
(run){
  EVENTS 3M;
  EVENT_GENERATION_MODE U;
  ME_SIGNAL_GENERATOR Comix;

  EVENT_OUTPUT HepMC_Short[SHERPA_QCD_2b2j];

  BEAM_1 2212; BEAM_ENERGY_1 7000;
  BEAM_2 2212; BEAM_ENERGY_2 7000;

  FRAGMENTATION=Off # disable hadronisation
  MI_HANDLER=None # disable multiple parton interactions

  SCF:=1; ### default scale factor
  SCALES VAR{SCF*H_T2/2};

  PDF_LIBRARY LHAPDFSherpa;
  PDF_SET NNPDF30_lo_as_0118_nf_4.LHgrid;
  PDF_SET_VERSION 0;

  MASSIVE[5] 1;
  MASS[5] 4.75;

}(run);
(processes){
  Process 93 93 -> 93 93 5 -5;
  Order_EW 0;
  End process;

}(processes);
(selector){
  PT 5 20 7000
  PT -5 20 7000
  PseudoRapidity 5 -3 3
  PseudoRapidity 5 -3 3

  PT 93 20 7000
  PseudoRapidity 93 -3 3
}(selector);
\end{verbatim}

\subsection {Background: QCD 4j}
\begin{verbatim}
(run){
  EVENTS 3M;
  EVENT_GENERATION_MODE U;
  ME_SIGNAL_GENERATOR Comix;

  EVENT_OUTPUT HepMC_Short[SHERPA_QCD_4j];

  BEAM_1 2212; BEAM_ENERGY_1 7000;
  BEAM_2 2212; BEAM_ENERGY_2 7000;

  FRAGMENTATION=Off # disable hadronisation
  MI_HANDLER=None # disable multiple parton interactions

  SCF:=1; ### default scale factor
  SCALES VAR{SCF*H_T2/2};

  PDF_LIBRARY LHAPDFSherpa;
  PDF_SET NNPDF30_lo_as_0118_nf_4.LHgrid;
  PDF_SET_VERSION 0;

  MASSIVE[5] 1;
  MASS[5] 4.75;

}(run);
(processes){
  Process 93 93 -> 93 93 93 93;
  Order_EW 0;
  End process;

}(processes);
(selector){
  PT 93 20 7000
  PseudoRapidity 93 -3 3
}(selector);
\end{verbatim}

%%%%%%%%%%%%%%%%%%%%%%%%%%%%%%%%%%
%%%%%%%%%%%%%%%%%%%%%%%%%%%%%%%%%%

\end{document}

