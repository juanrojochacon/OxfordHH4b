
\section{Analysis strategy}
\label{sec:analysis}

In section we describe our analysis strategy, and in particular
the classification of each individual event into
different categories according to its topology.
%
First of all, we discuss the settings and nomenclature
for jet clustering, and the strategy for $b$-tagging.
%
Then we discuss the categorisation of events between different
topologies and how to prioritise among them.
%
Finally, we motivate our choice of analysis cuts by comparing signal and background
for representative kinematic distributions.

\subsection{Jet reconstruction}

After the  parton shower, final state particles
are clustered using the
jet reconstruction algorithms
obtained from
{\tt FastJet}~\cite{Cacciari:2011ma,Cacciari:2005hq},
{\tt v3.1.0}.
%
We will use different types of jet definitions:
\begin{itemize}
\item {\it Small-$R$ jets}.

  These are jets  reconstructed with the
  anti-$k_T$ clustering algorithm~\cite{Cacciari:2008gp} with $R=0.4$ radius.
  %
  These small-$R$ jets are required
  to have transverse momentum $p_T \ge 40$~GeV
  and pseudo-rapidity $|\eta|<2.5$, within the central 
  acceptance of ATLAS and CMS, the region
  where $b$-tagging is possible.

\item {\it Large-$R$ jets}.

  These jets are also constructed with the
  anti-$k_T$ clustering algorithm, now using a $R=1.0$ radius.
  %
  Large-$R$ jets are required to have
  $p_T \ge 200$~GeV and lie in a pseudo-rapidity region of
  $|\eta|<2.0$.
  %
  The more restrictive range  in pseudo-rapidity
  as compared to the small-$R$ jets
  is motivated by mimicking the  experimental requirements
  in ATLAS and CMS
  related to the track-jet based calibration~\cite{Aad:2014bia,ATLAS:2012kla}.

  In addition to the basic $p_T$ and $\eta$
  acceptance requirements, large-$R$ jets should also
  satisfy the  BDRS mass-drop tagger (MDT)(~\cite{Butterworth:2008iy}
  conditions.
  %
  We use the {\tt FastJet} default
  parameters of  $\mu = 0.67$ and $y_{\textrm{cut}}= 0.09$.
  %
  Before applying the MDT, the large-$R$ jet
  constituents are reclustered with the Cambridge/Aachen (C/A)
  algorithm~\cite{Dokshitzer:1997in,Wobisch:1998wt}
  with $R=1.2$.

  
\item {\it Small-$R$ subjets}.

  These subjets are constructed by reclustering the constituents
  of a large-$R$ jet, again with  anti-$k_T$ algorithm,
  but this time with a smaller radius parameter, namely
  $R=0.3$.
 %
  These small-$R$ subjets will be the main input for
  $b$-tagging in the boosted category.
  %
  They are required to satisfy $p_T > 50$~GeV and $|\eta|<2.5$.
  %
\end{itemize}

In addition, we have
also explored the possible improvements
in the analysis from the use
of variable-$R$ jets~\cite{Krohn:2009zg}, but found that these made
a small difference as compared to fixed-$R$ jets.
%
This is not unexpected, since variable-$R$ jets are more suitable when
the degree of boost of the final state being reconstructed can span
a wide range~\cite{Cacciari:2008gd}, such as in heavy resonance searches~\cite{Aad:2015fna},
where where the degree of boost of the final state
is {\it a priori} unknown.


\subsection{Tagging of $b$-jets}
\label{sec:btagging}

It is clear that
optimisation of the $b$-tagging capabilities of the
LHC experiments is an essential ingredient to improve
the prospects for the observation of Higgs pair production in the
$b\bar{b}b\bar{b}$ final state.
%
In this work, we adopt
a $b$-tagging strategy along the lines
of current ATLAS performance~\cite{Aad:2013gja,Aad:2015ydr},
though differences with respect to
the corresponding CMS strategy~\cite{Khachatryan:2011wq,Chatrchyan:2012jua}
should not affect qualitatively our results.
%
For each type of jet definition used, a different
$b$-tagging strategy is adopted:

\begin{itemize}

\item {\it Small-$R$ jets}.

  %
  If a small-$R$ jet has at least one $b$-quark among their constituents,
  it will be tagged as a $b$-jet with probability $f_b$.
  %
  In order to be considered in the $b$-tagging algorithm,
  $b$-quarks inside the small-$R$ jet
  should satisfy $p_T \ge 15$ GeV.
  %
  The probability of tagging a jet is not modified
  if more than one $b$-quark is found among the jet constituents.


  
  If no $b$-quarks are found among the constituents
  of this jet, it can be still be tagged as a $b$-jet with
  a mistag rate of $f_l$, unless a charm quark is present instead,
  and in this case the mistag rate is $f_c$.
  %
  Only jets which contain at least one (light or charm)
  constituent
  with $p_T \ge 15$ GeV will lead to fake $b$-tags.

  
  We only attempt to $b$-tag only the four (two) hardest small-$R$ jets
  in the resolved (intermediate) category.
  %
  Trying to $b$-tag all the
  small-$R$ jets that satisfy the acceptance cuts worsens the
  overall performance
  due to combinatorics.

  \item {\it Large-$R$ jets}.

    Large-$R$ jets are $b$-tagged by
    ghost-associating anti-$k_T$ $R=0.3$
    subjets to the original large-$R$
    jets~\cite{Cacciari:2007fd,Aad:2013gja,
      ATLAS-CONF-2014-004,Aad:2015uka}.
    %
    A large-$R$ jet will considered $b$-tagged if both
    the leading and subleading AK03 subjets, where the ordering
    is done in the subjet $p_T$, are both individually $b$-tagged,
    with the same criteria as the small-$R$ jets.

    
    As in the case
    of small-$R$ jets, we only attempt to $b$-tag the two leading subjets,
    else one finds a degradation of the
    signal significance due to combinatorics.

    %
    Therefore, a large-$R$ jet where the two leading
    subjets have at least one $b$-quark will be tagged
    with probability $f_b^2$, if only one of the two leading
    subjets has a $b$-quark, then the tagging probability is
    $2f_bf_l$, and if none of the two subjets include $b$-quarks,
    the mistag rate will be $f_l^2$.


\end{itemize}

Concerning the values of the $b$-tag probability, $f_b$, and
the $b$-mistag probability of light and charm jets, $f_l$ and  $f_c$
respectively,
we used the values $f_b=0.8$, $f_l=0.01$
and  $f_c=0.1$.


\subsection{Event categorisation}
\label{sec:categorisation}

The basic idea of the present analysis is similar to the
scale-invariant resonance tagging introduced
in Ref.~\cite{Gouzevitch:2013qca}.
%
Rather than restricting to a specific event topology,
we aim to consistently combine the information from
the three possible event topologies: boosted, intermediate and
resolved.
%
The optimal cuts for each category will be determined
separately, and then
a exclusive categorisation is defined starting from the category
with the highest signal significance.
%
An advantage of our approach is that the analysis strategy
is robust under variations of
the underlying production model of Higgs pairs is modified,
for instance in the case of
BSM dynamics, which can substantially increase
the degree of boost of the final state.
%
These three categories are defined as follows:
\begin{itemize}
\item {\it Boosted category}.

  An event which
  contains at least two large-$R$ jets, with the two leading
 being $b$-tagged.
 %
 Each of these two large-$R$ $b$-tagged jets are is this
 a candidate
 to contain the decay products of a Higgs boson.

\item {\it Intermediate category}.

  An event with only one large-$R$ $b$-tagged jet, which
  is assigned to be the leading Higgs candidate.
  %
  In addition, we require at least two small-$R$ $b$-tagged jets,
  which must be separated with respect to the large-$R$ jet
  by $\Delta R\ge 1.0$.
  %
    
  The subleading Higgs boson candidate is then reconstructed
  by selecting the two $b$-tagged small-$R$ jets that minimize the difference
  between the invariant mass of the large-$R$ jet
  with the invariant mass of the dijet obtained
  from the sum of the two small-$R$ jets.
  
\item {\it Resolved category}.

  
 An event with at least
  four $b$-tagged small-$R$ jets.
  %
  The two Higgs candidates are reconstructed out of the
  leading four small-$R$ jets in the event
  by considering all possibilities of forming two pairs of jets
  and then choosing the configuration that minimizes the relative difference of
  dijet masses.
  %

  
\end{itemize}


Once a Higgs boson candidate has been identified,
its invariant mass is required to lie within a fixed window
around the nominal Higgs boson mass.
%
In particular we require the condition
\be
\label{higgsmasswindow}
|m_{h,j} - 125| < 40~{\rm GeV} \, ,\, j=1,2 \, ,
\ee
where $m_{h,j}$ is the invariant mass of each of the two reconstructed  Higgs candidates.
%
This cut is substantially looser as the corresponding
cut used in the typical ATLAS and CMS $h\to b\bar{b}$
analyses~\cite{Aad:2012gxa,Chatrchyan:2013zna}: the motivation
for this choice is that the MVA will take care of the optimisation of these
and other cuts that are not explicitly performed.
%
Only events where the two Higgs candidates satisfy
Eq.~(\ref{higgsmasswindow}) are classified as signal events.
%

These three categories are not exclusive:
a given event can be assigned to more than one category, for
example, satisfying the requirements of both the intermediate
and resolved
categories at the same time.
%
The exception is the boosted and intermediate categories, which have
orthogonal jet selection requirements.
%
The separate optimisation of the three categories needs to be performed in
a inclusive analysis, and then we introduce a
 priorisation starting with the category of highest signal significance.


 This is achieved as follows.
 %
First of all we perform an inclusive analysis, and optimise the
signal significance
$S/\sqrt{B}$ in each of the three categories separately, including
the MVA.
%
We find that the category with higher significance turns out to be the boosted,
followed by the intermediate and the resolved ones, the two latter
with similar significance.
%
Therefore for the final analysis we turn to an exclusive approach:
if an event satisfies the boosted requirements, it is kept for
these category, else we check if it suits the intermediate
requirements, and failing also this, the event is attempted to
be classified in the
resolved category.
%
The resulting exclusive event samples are then separately processes
through the MVA.
%
This allows a consistent combination
of the significance of the three event categories.







\subsection{Optimisation of kinematical cuts}

We now motivate the values of
kinematical cuts applied to the different categories.
%
These cuts have been separately
optimised in an inclusive analysis, since it is only
once we have determined
the dominant category that the analysis is made exclusive.
%
Here we show the various kinematic distributions 
before the corresponding cut has been applied.
%
For illustration purposes,
only a representative subset of
all available kinematic distributions will be shown here.
%
For simplicity,
we restrict to the discussion of the
kinematic distributions of the boosted
and resolved categories.


First of all,
in Fig.~\ref{fig:cutplots1} we show
the $p_T$ distribution of the 
of the
  leading and subleading large-$R$ jets in the boosted category, comparing
  the shapes of the distribution in signal and background events.
  %
  Distributions have been normalized to their total integral.
  %
  We see that the background distribution
  falls off rather steeply than the signal at large $p_T$
  values, therefore
  providing evidence that exploiting the boosted region will be
  advantageous to enhance the signal over background ratio.
  %
  This is true both for the leading and the subleading
  large-$R$ jets.
  %
  These distributions justify our choice of $p_T \ge 200$ GeV
  for the large-$R$ jets in the boosted category.
  

%%%%%%%%%%%%%%%%%%%%%%%%%%%%%%%%%%%%
\begin{figure}[t]
\begin{center}
 \includegraphics[width=0.48\textwidth]{plots/pt_H0_boost_C1.pdf}
 \includegraphics[width=0.48\textwidth]{plots/pt_H1_boost_C1.pdf}
\caption{\small  The $p_T$ distribution of the
  leading (left plot) and
  subleading (right plot) large-$R$ jets in the boosted category.
  %
  We comparing
  the shapes of the distribution in signal and background events,
  where both
  distributions have been normalized to their total integral.
}
\label{fig:cutplots1}
\end{center}
\end{figure}
%%%%%%%%%%%%%%%%%%%%%%%


In the boosted category we require the two
leading subjets of the large-$R$ jet to be relatively
hard, in particular that should satisfy $p_T \ge $ 50 GeV.
%
To motivate this cut, in Fig.~\ref{fig:cutplots22}
we show the distribution in $p_T$ of the leading
and subleading sub-jets in the leading large-$R$ jet in events
of the boosted category.
%
It is clear from the comparison that the subjet $p_T$ spectrum is
relatively harder in the signal with respect to the background,
motivating our choice of subjet $p_T$ cut.


%%%%%%%%%%%%%%%%%%%%%%%%%%%%%%%%%%%%
\begin{figure}[t]
\begin{center}
 \includegraphics[width=0.48\textwidth]{plots/pt_leadSJ_fj1_noPU.pdf}
 \includegraphics[width=0.48\textwidth]{plots/pt_subleadSJ_fj1_noPU.pdf}
\caption{\small  The $p_T$ distribution of the
  leading (left plot) and
  subleading (right plot) AKT03 subjets within the leading
  Higgs candidate in the boosted category.
  %
  We comparing
  the shapes of the distribution in signal and background events,
  where both
  distributions have been normalized to their total integral.
}
\label{fig:cutplots22}
\end{center}
\end{figure}
%%%%%%%%%%%%%%%%%%%%%%%



In the resolved category,
it is important to understand the $p_T$ distribution
of the four leading small-$R$ jets of the event.
%
As noted in~\cite{deLima:2014dta}
given that in general the boost from the Higgs decays is moderate,
the two subleading jets might be not too hard, and thus it
should be important to ensure that our $p_T^{\rm min}$ cut
is not too strong.
%
This also has important implications for the experimental
trigger requirements in this category.
%
In Fig.~\ref{fig:cutplots23}
we show the distribution in $p_T$ of the four leading
small-$R$ jets in signal and background events.
%
We see that the third and four leading jets are relatively soft.
%
Therefore, we can conclude that....

%%%%%%%%%%%%%%%%%%%%%%%%%%%%%%%%%%%%
\begin{figure}[t]
\begin{center}
 \includegraphics[width=0.48\textwidth]{plots/pt_smallRjets_res_noPU.pdf}
 \includegraphics[width=0.48\textwidth]{plots/eta_smallRjets_res_noPU.pdf}
 \caption{\small Same as Fig.~\ref{fig:cutplots23}, now for the
   $p_T$ and rapidity distributions of the small-$R$
   jets in the boosted category.
}
\label{fig:cutplots23}
\end{center}
\end{figure}
%%%%%%%%%%%%%%%%%%%%%%%


Another important distribution is the rapidity of the large-$R$
(in the boosted category) and the small-$R$ (in the resolved
category) jets.
%
We want to understand by how much the signal efficiency is reduced
by the restriction of the central region, and how would
the signal yield be enhanced if tracking could be extended
to forward rapidities in ATLAS and CMS.
%
Also, it is useful to show the distribution in rapidity of the
individual jets in the resolved category.
%

Next, in Fig.~\ref{fig:mHHinv} we show the invariant mass
of the leading reconstructed Higgs candidates, before the Higgs mass window
cut Eq.~(\ref{higgsmasswindow})
  is applied, for the resolved and boosted categories.
%
The signal distribution is of course peaked at the
nominal Higgs mass of $m_h=125$ GeV.
%
In the boosted category, the background shows no particular
structure, while remarkable in the resolved category the background
also shows a peak-like structure, artificially induced by the
acceptance kinematical cuts on the jets.
%
Note that the applied smearing, which mimics detector resolution effects,
leads to a rather broad distribution which reduces the usefulness
of the Higgs mass window cut.
%
While our Higgs mass window cut is relatively loose,
information of the different shape of the $m_{h}$
distribution will still be exploited by the MVA.
%

%%%%%%%%%%%%%%%%%%%%%%%%%%%%
\begin{figure}[t]
\begin{center}
  \includegraphics[width=0.48\textwidth]{plots/m_H0_res_C1.pdf}
  \includegraphics[width=0.48\textwidth]{plots/m_H0_boost_C1.pdf}
  \caption{\small The invariant mass distribution of the leading
    Higgs candidates on the resolved (left plot) and boosted (right
    plot) categories. 
}
\label{fig:mHHinv}
\end{center}
\end{figure}
%%%%%%%%%%%%%%%%%%%%%%%


Another important kinematical distribution of this process is the invariant mass
of the di-Higgs system.
%
First of all because this is a direct measure of the boost of the system,
and also because in many BSM scenarios this distribution can be substantially
modified as compared to the SM case, for example in the presence
of certain dimension-6 EFT operators~\cite{Azatov:2015oxa}.
%
Indeed, one important advantage of the $4b$ final state for
di-Higgs production is that it significantly increases the reach
in $m_{hh}$ as compared to other channels with smaller branching
ration,
such as $2b2\gamma$.
%
With this motivation, we show in
Fig.~\ref{fig:mhh} the invariant mass distribution of the
reconstructed Higgs pairs,
comparing the resolved category (left) with the boosted category (right).


In the resolved case, we see that the distribution
in $m_{hh}$ is rather harder for the signal as for the background,
and thus one expects that cutting in $m_{hh}$ would help signal
discrimination: we will verify this with the help of the MVA.
%
For the boosted category the trend of the $m_{hh}$ distribution
is different because of the jet selection cuts, with the
distribution now peaking at higher values.
%
In this case signal and background distributions
look reasonably similar.
%
Note that at parton-level the $m_{hh}$ distribution as a kinematical
cut-off at $m_{hh}^{\rm min}=250$ GeV, which is modified by both
shower effects and by detector resolution effects.
%
While we don not cut in this quantity, it is
one of the inputs for the MVA to help to improve signal discrimination.

%%%%%%%%%%%%%%%%%%%%%%%%%%%%
\begin{figure}[t]
\begin{center}
  \includegraphics[width=0.48\textwidth]{plots/m_HH_res_C1.pdf}
  \includegraphics[width=0.48\textwidth]{plots/m_HH_boost_C1.pdf}
  \caption{\small Left plot: the invariant mass distribution of the Higgs
    pair candidates, $m_{hh}$, comparing signal and background events,
    in the resolved category.
    %
    Right plot: same for the boosted category.
}
\label{fig:mhh}
\end{center}
\end{figure}
%%%%%%%%%%%%%%%%%%%%%%%


Another interesting distribution is the transverse momentum of
the di-Higgs system.
%
In Fig.~\ref{fig:pthh} we show the $p_T^{hh}$
distribution
for the resolved and boosted categories.
%
Again we see that the background has a steeper $p_T^{hh}$ distribution
that the signal, in both categories, thus this variable
should provide some additional discrimination power, and therefore
we will use it as another of the MVA inputs.
%
Note that in our LO simulation this distribution is generated purely
by the parton shower - a more refined calculation would require
either the matching with higher-multiplicity matrix elements~\cite{Maierhofer:2013sha} or
the full NLO calculation~\cite{Frederix:2014hta}, to treat properly the first hard emission.
%
Nevertheless, the MVA shows only limited sensitivity to this variable, so its
modelling appears not to be crucial in our case.

%%%%%%%%%%%%%%%%%%%%%%%%%%%%
\begin{figure}[t]
\begin{center}
  \includegraphics[width=0.48\textwidth]{plots/pt_HH_res_C1.pdf}
  \includegraphics[width=0.48\textwidth]{plots/pt_HH_boost_C1.pdf}
  \caption{\small Same as Fig.~\ref{fig:mhh} for the transverse momentum
    distribution of the di-Higgs system $p_T^{hh}$.
}
\label{fig:pthh}
\end{center}
\end{figure}
%%%%%%%%%%%%%%%%%%%%%%%


\subsection{PU subtraction with {\tt SoftKiller}}

To study the impact of PU in our analysis,
Minimum Bias (MB) events,
including Multiple Parton Interactions (MPI), have been generated
with {\tt Pythia8}, and
superimposed to the signal
and background samples described in Sect.~\ref{mcgeneration}.
%
We have explored two scenarios for the amount of PU expected
at the HL-LHC, one with a mean number of
PU vertices per bunch crossing of $\la n_{\rm PU}\ra=80$, and another
with $\la n_{\rm PU}\ra=150$.
%
In order to subtract the PU, a number of techniques
have been developed
recently~\cite{Cacciari:2009dp,TheATLAScollaboration:2013pia,Butterworth:2008iy,Cacciari:2007fd,Krohn:2009th,Krohn:2013lba,Ellis:2009me,Bertolini:2014bba,Cacciari:2014gra,Cacciari:2014jta,Berta:2014eza,Larkoski:2014wba}.\footnote{
These techniques have also important applications in the subtraction
of the UE/MPI contamination for jet reconstruction
in heavy ion collisions~\cite{Cacciari:2010te}.
}
%
In this work, PU  will be subtracted by means
of the the {\tt SoftKiller} (SK)
method~\cite{Cacciari:2014gra}, as implemented in {\tt FastJet}.
%

The idea underlying {\tt SoftKiller} is based on eliminating particles
below a given cut-off in their transverse momentum, $p_T^{\rm (cut)}$, whose
value is dynamically determined in a way that makes the event-wide
transverse-momentum flow density $\rho$ vanish.
%
This $p_T$ flow density is defined as
\be
\rho\equiv{\rm median}_i \Bigg\{ \frac{p_{Ti}}{A_i}\Bigg\} \, ,
\ee
where the median is computed over all the patches $i$ with area
$A_i$ and transverse momentum $p_{Ti}$ in which the $\lp \eta,\phi\rp$ plane
is partitioned.
%
From its definition in terms of the median,
we observe that the value of $p_T^{(\rm cut)}$
will be dynamically raised until half of the patches have $\rho=0$.
%
The size and number of these patches is a free parameter of the algorithm -
here we will use square patches with length $a=0.4$.
%
We restrict ourselves to the central rapidity region,
$|\eta| \le 2.5$, for the estimation of the
$p_T$ flow density $\rho$.
%
In this analysis, {\tt SoftKiller} method is applied
to particles at the end of the parton shower, before
jet clustering.

To validate PU subtraction,
we now compare different kinematical distributions
in the case without PU and in the case
with PU subtracted with {\tt SoftKiller}.
%
Then we will also compare some specific distributions
also in the case where PU has not been subtracted.
%
We find that
{\tt SoftKiller} exhibits a reasonable
performance for the PU subtraction.
%
We  also obtain that the boosted category is less affected
by PU than the resolved category, as expected from the higher
$p_T$ threshold in the corresponding selection cuts.


First of all, let us validate both the implementation of our
simulation of PU, as well as the SK subtraction, by comparing the
invariant mass distributions of Higgs candidates in signal
events in the resolved category.
%
In Fig.~\ref{fig:PUvalidation} we show three curves: without PU,
with PU $\la n_{PU}\ra=80$ but without any subtraction, and the
same but now with the SK subtraction.
%
As expected, if PU is not subtracted there is a large shift in the Higgs mass
peak, by about 40 GeV.
%
Once SK subtraction is performed, we recover a distribution much closer
to the original ones, with only a small shift of $\simeq 5$ GeV in the mass
distribution.
%
As we will see below, the performance of SK is even better in the boosted category,
since the effects of PU are mitigated for high $p_T$ jets.

%%%%%%%%%%%%%%%%%%%%%%%%
\begin{figure}[t]
  \begin{center}
  \includegraphics[width=0.68\textwidth]{plots/m_htot_res_signal_PUnoSK.pdf}
    \caption{\small
    The invariant mass distributions of Higgs candidates in signal
events, in the resolved category, comparing the results without PU,
with PU $\la n_{PU}\ra=80$ but without any subtraction, and the
corresponding results now with SK subtraction.
}
\label{fig:PUvalidation}
\end{center}
\end{figure}
%%%%%%%%%%%%%%%%%%%%%%%

Next, we compare more
 signal distributions with
and without PU, with $\la n_{\rm PU}\ra=80$ in the
former
case (and SK subtraction).
%
In Fig.~\ref{fig:m_H_PU} we show the invariant mass distribution
of the leading and subleading Higgs candidates, corresponding to both
the boosted
and resolved categories.
%
These distributions are plotted after the $b$-tagging, that is,
before they are used as input to the MVA.
%
As we can see, in the boosted category, the residual effects of PU
after the {\tt SoftKiller} subtraction are rather mild,
with the position of the Higgs mass peaks essentially
unchanged, and only a moderate distortion of the
distribution found.
%
The effects are more important for the resolved category, where PU
shifts the Higgs peak by an amount $\Delta m_h \simeq 5$ GeV.

%%%%%%%%%%%%%%%%%%%%%%%%
\begin{figure}[t]
  \begin{center}
      \vspace{-1cm}
      \includegraphics[width=0.49\textwidth]{plots/m_H0_res_comp.pdf}
      \includegraphics[width=0.49\textwidth]{plots/m_H1_res_comp.pdf}
      \includegraphics[width=0.49\textwidth]{plots/m_H0_bst_comp.pdf}
      \includegraphics[width=0.49\textwidth]{plots/m_H1_bst_comp.pdf}
  \caption{\small
    Comparison of the invariant mass distributions of the leading (left plots)
    and subleading (right plots) Higgs candidates in the resolved
    (upper plots) and boosted (lower plots) categories,
    both without PU and with
    PU, $\la n_{PU}\ra=80$, subtracted with {\tt SoftKiller}.
}
\label{fig:m_H_PU}
\end{center}
\end{figure}
%%%%%%%%%%%%%%%%%%%%%%%

Next we compare the transverse momentum of the leading Higgs
candidate, $p_t^{h_1}$ and the invariant mass of the di-Higgs system
$m_{hh}$, in Fig.~\ref{fig:mHH_PU}, both for the boosted and
for the resolved categories.
%
In the case of the $p_T$ distribution, the differences between the selection
criteria for the resolved
and boosted categories is reflected in the rightward shift of the latter.
%
The effect of PU is negligible in the boosted case, and small
in the resolved case, except for large $p_T$ values.
%
For the case of the $m_{hh}$ distribution, similar conclusions
apply.
%
These comparisons validate the PU subtraction strategy
adopted in this work.


%%%%%%%%%%%%%%%%%%%%%%%%
\begin{figure}[t]
  \begin{center}
    \vspace{-1cm}
  \includegraphics[width=0.49\textwidth]{plots/pt_H0_C2_res_comp.pdf}
  \includegraphics[width=0.49\textwidth]{plots/pt_H0_C2_bst_comp.pdf}
  \includegraphics[width=0.49\textwidth]{plots/m_HH_C2_res_comp.pdf}
  \includegraphics[width=0.49\textwidth]{plots/m_HH_C2_bst_comp.pdf}
  \caption{\small
    Comparison of the transverse momentum $p_T^h$ of the leading
    Higgs candidate (upper plots) and of the invariant mass $m_{hh}$
    of the di-Higgs system (lower plots) in the resolved
    (left plots) and boosted (right plots) categories,
    without PU and with $\la n_{PU}\ra=80$ subtracted with {\tt SoftKiller}.
}
\label{fig:mHH_PU}
\end{center}
\end{figure}
%%%%%%%%%%%%%%%%%%%%%%%

We can also assess the impact of PU in representative
substructure variables
used as input to the MVA in the boosted category.
%
In particular we consider the subjetiness variable,
$\tau_{21}$, Eq.~(\ref{eq:tau21}), and the ratio
of energy correlation functions, $D_2^{(\beta)}$,
Eq.~(\ref{eq:d2}),
corresponding to the leading Higgs candidate.
%
This comparison is illustrated in Fig.~\ref{fig:Substructure_PU}.
%
As can be seen, these substructure variables, which
as demonstrated before carry a substantial
discrimination power, are relatively unaffected by PU.
%
Recall that the $D_2^{(\beta)}$ variable is
explicitly constructed~\cite{Larkoski:2013eya}
to be resilient
with respect to PU contamination.
%
Therefore, we do not expect a major loss of discrimination
power due to PU effects in our analysis.
%

%%%%%%%%%%%%%%%%%%%%%%%%
\begin{figure}[t]
  \begin{center}
  \includegraphics[width=0.49\textwidth]{plots/D2_h0_bst_comp.pdf}
  \includegraphics[width=0.49\textwidth]{plots/tau21_h0_bst_comp.pdf}
   \caption{\small
     Comparison of the substructure variables $D_2^{(\beta)}$ (left)
     and $\tau_{21}$ (right)
     for the leading Higgs candidate in the boosted category,
   without PU and with $\la n_{PU}\ra=80$ subtracted with {\tt SoftKiller}.
}
\label{fig:Substructure_PU}
\end{center}
\end{figure}
%%%%%%%%%%%%%%%%%%%%%%%

It is also interesting to quantify how
the relative differences between
signal over background distributions are affected in 
the presence of PU.
%
Considering first of all the boosted category,
in Fig.~\ref{fig:signal-vs-back-boosted} we compare
various kinematical distributions for signal and background events,
     with and without PU: the invariant mass, the $p_T$,
     the 2-jettiness $\tau_{21}$, and $\sqrt{d_{12}}$,
     the $k_T$ splitting scale, Eq.~(\ref{eq:ktsplitting}),
     all corresponding to
     the leading Higgs candidate.
     %
      We observe that the most of the relevant
      qualitative differences between signal
      and background distributions are maintained in the presence of PU.
      %
      This is specially clear for the substructure variables, which
      show a similar discrimination power both with and without
      PU.
     

%%%%%%%%%%%%%%%%%%%%%%%%
\begin{figure}[t]
  \begin{center}
  \includegraphics[width=0.49\textwidth]{plots/m_h0_bst_comp_back.pdf}
  \includegraphics[width=0.49\textwidth]{plots/pt_h0_bst_comp_back.pdf}
   \includegraphics[width=0.49\textwidth]{plots/tau21_h1_bst_comp_back.pdf}
  \includegraphics[width=0.49\textwidth]{plots/split12_h0_bst_comp_back.pdf}
   \caption{\small
     Comparison of kinematical distributions, in
     the boosted category, for signal and background events
     with and without PU: the invariant mass,  $p_T$,
     and the substructure variables $\tau_{21}$ and $\sqrt{d_{12}}$
    for the leading Higgs candidate.
     %
 }
\label{fig:signal-vs-back-boosted}
\end{center}
\end{figure}
%%%%%%%%%%%%%%%%%%%%%%%



Now we perform a similar comparison this time for
the resolved category.
%
In Fig.~\ref{fig:signal-vs-back-resolved} we compare
the kinematical distributions for signal and background events,
     with and without PU, for the invariant mass and the $p_T$ of the leading
     Higgs candidate.
     %
     Also in this
     case the PU-subtracted background distributions appear reasonably close
     to their no PU counterparts.
     %
     Therefore, these results
     suggest that the broad pattern of the signal over background
     discrimination provided by the MVA in Sect.~\ref{sec:mva},
     based on the corresponding differences between kinematical
     distributions,
will
     be maintained in the presence of PU.
     %
     In the next section we verify this expectation.


%%%%%%%%%%%%%%%%%%%%%%%%
\begin{figure}[t]
  \begin{center}
   \includegraphics[width=0.49\textwidth]{plots/m_h0_res_comp_back.pdf}
  \includegraphics[width=0.49\textwidth]{plots/pt_h0_res_comp_back.pdf}
     \caption{\small
       Same as Fig.~\ref{fig:signal-vs-back-boosted} for the resolved category,
       this time without the jet substructure variables.
}
\label{fig:signal-vs-back-resolved}
\end{center}
\end{figure}
%%%%%%%%%%%%%%%%%%%%%%%
