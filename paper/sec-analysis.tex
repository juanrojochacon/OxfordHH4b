
\section{Analysis strategy}
\label{sec:analysis}

In this section we describe our analysis strategy.
%
First of all we discuss the settings
for jet clustering and the strategy for jet $b$-tagging.
%
Following this we discuss the categorisation of events into different
topologies, and how the different topologies may be prioritised.
%
We motivate our choice of analysis cuts by comparing signal and background
distributions for representative kinematic variables.
%
Finally, we describe the simulation of PU and validate
the PU subtraction strategy.

\subsection{Jet reconstruction}

After the parton shower, final state particles
are clustered using the
jet reconstruction algorithms
of
{\tt FastJet}~\cite{Cacciari:2011ma,Cacciari:2005hq},
{\tt v3.1.0}.
%
Here we use the following jet definitions:
\begin{itemize}
\item {\it Small-$R$ jets}.

  These are jets  reconstructed with the
  anti-$k_T$ clustering algorithm~\cite{Cacciari:2008gp} with $R=0.4$ radius.
  %
  These small-$R$ jets are required
  to have transverse momentum $p_T \ge 40$~GeV
  and pseudo-rapidity $|\eta|<2.5$, within the central 
  acceptance of ATLAS and CMS, and therefore within the region
  where $b$-tagging is possible.

\item {\it Large-$R$ jets}.

  These jets are also constructed with the
  anti-$k_T$ clustering algorithm, now using a $R=1.2$ radius.
  %
  Large-$R$ jets are required to have
  $p_T \ge 200$~GeV and lie in a pseudo-rapidity region of
  $|\eta|<2.0$.
  %
  The more restrictive range  in pseudo-rapidity
  as compared to the small-$R$ jets
  is motivated by mimicking the  experimental requirements
  in ATLAS and CMS
  related to the track-jet based calibration~\cite{Aad:2014bia,ATLAS:2012kla}.

  In addition to the basic $p_T$ and $\eta$
  acceptance requirements, large-$R$ jets should also
  satisfy the  BDRS mass-drop tagger (MDT)~\cite{Butterworth:2008iy}
  conditions, where the {\tt FastJet} default
  parameters of  $\mu = 0.67$ and $y_{\textrm{cut}}= 0.09$ are used.
  %
  Before applying the BDRS tagger, the large-$R$ jet
  constituents are reclustered with the Cambridge/Aachen (C/A)
  algorithm~\cite{Dokshitzer:1997in,Wobisch:1998wt}
  with $R=1.2$.

  
\item {\it Small-$R$ subjets}.

  These subjets are constructed by reclustering the constituents
  of a large-$R$ jet, again with the
  anti-$k_T$ algorithm,
  but this time with a smaller radius parameter, namely
  $R=0.3$~\cite{Aad:2015uka}. They are required to satisfy $p_T > 50$~GeV and $|\eta|<2.5$.
 %
  These small-$R$ subjets will be the main input for
  $b$-tagging in the boosted category.
  %
\end{itemize}

For the   boosted and intermediate categories,
which involve the use of large-$R$ jets,
we use jet substructure variables~\cite{Salam:2009jx,Aad:2013gja} to
improve the significance of the discrimination between signal and background
events in the MVA.
%
In particular we  consider the following jet
substructure variables:
\begin{itemize}
\item The $k_T$-splitting scale~\cite{Butterworth:2002tt,Butterworth:2008iy}.

  This variable is obtained by reclustering the constituents of a jet with the
  $k_t$ algorithm~\cite{Ellis:1993tq},
  which usually clusters last the harder constituents, and then
  taking the $k_t$ distance measure between the two subjets at the final stage of the recombination
  procedure,
  \be
  \label{eq:ktsplitting}
\sqrt{d_{12}} \equiv {\rm min}\lp p_{T,1},p_{T,2}\rp \cdot \Delta R_{12} \, .
\ee
with $p_{T,1}$ and $p_{T,2}$ the transverse momenta of the two subjets merged
in the final step of the clustering, and $\Delta R_{12}$ the corresponding
angular separation.
  
\item The ratio of 2-to-1 subjettiness $\tau_{12}$~\cite{Thaler:2010tr,Thaler:2011gf}.

  The $N$-subjettiness variables $\tau_N$ are defined by clustering the constituents
  of a jet with the exclusive $k_t$ algorithm~\cite{Catani:1993hr}
  and requiring that $N$ subjets are found,
  \be
  \tau_N \equiv \frac{1}{d_0} \sum_k p_{T,k}\cdot {\rm min}\lp \delta R_{1k}, \ldots,
  \delta R_{Nk}\rp \, , \qquad d_0\equiv \sum_k p_{T,k}\cdot R \, ,
  \ee
  where $p_{T,k}$ is the $p_T$ of the constituent particle $k$ and $\delta R_{ik}$ the distance from
  subjet $i$ to constituent $k$.
  %
  In this work we use as input to the MVA the ratio of 2-subjettiness to 1-subjettiness, namely
  \be
  \label{eq:tau21}
\tau_{21} \equiv \frac{\tau_2}{\tau_1} \, ,
  \ee
  which provides good discrimination 
  between QCD jets and jets arising from the decay of
  a heavy resonance.
  
\item The ratios of energy correlation functions (ECFs)  $C^{(\beta)}_2$~\cite{Larkoski:2013eya} and
  $D_2^{(\beta)}$~\cite{Larkoski:2014gra}.

  The ratio of energy correlation functions $C_2^{(\beta)}$ is defined as
  \be
  \label{eq:c2}
C_2^{(\beta)} \equiv \frac{ {\rm ECF}(3,\beta) {\rm ECF}(1,\beta)}{\lc {\rm ECF}(2,\beta)\rc ^2} \, ,
\ee
while $D_2^{(\beta)}$ is instead defined as a double ratio of ECFs, that is
\be
e_3^{(\beta)}\equiv \frac{ {\rm ECF}(3,\beta)}{\lc {\rm ECF}(1,\beta)\rc^3} \, , \quad
  e_2^{(\beta)}\equiv \frac{ {\rm ECF}(2,\beta)}{\lc {\rm ECF}(1,\beta)\rc^2} \, , \quad
  \label{eq:d2}
D_2^{(\beta)} \equiv \frac{ e_3^{(\beta)})}{\lp e_2^{(\beta)} \rp^3} \, .
\ee
The energy correlation functions ${\rm ECF}(N,\beta)$ are defined
  in~\cite{Larkoski:2013eya} with the motivation that $(N+1)$-point correlators
  are sensitive to $N$-prong substructure.
  %
  The free parameter $\beta$ is set to a value of $\beta=2$,
  as recommended by the authors of Refs.~\cite{Larkoski:2013eya,Larkoski:2014gra}.
\end{itemize}



\subsection{Tagging of $b$-jets}
\label{sec:btagging}

In this analysis we adopt
a $b$-tagging strategy along the lines
of current ATLAS performance~\cite{Aad:2013gja,Aad:2015ydr},
though differences with respect to
the corresponding CMS
settings~\cite{Khachatryan:2011wq,Chatrchyan:2012jua}
do not modify qualitatively our results.
%
For each jet definition described above, a different
$b$-tagging strategy is adopted:

\begin{itemize}

\item {\it Small-$R$ jets}.

  %
  If a small-$R$ jet has at least one $b$-quark among their constituents,
  it will be tagged as a $b$-jet with probability $f_b$.
  %
  In order to be considered in the $b$-tagging algorithm,
  $b$-quarks inside the small-$R$ jet
  should satisfy $p_T \ge 15$ GeV~\cite{Aad:2015ydr}.
  %
  The probability of tagging a jet is not modified
  if more than one $b$-quark is found among the jet constituents.


  
  If no $b$-quarks are found among the constituents
  of this jet, it can be still be tagged as a $b$-jet with
  a mistag rate of $f_l$, unless a charm quark is present instead,
  and in this case the mistag rate is $f_c$.
  %
  Only jets that contain at least one (light or charm)
  constituent
  with $p_T \ge 15$ GeV can induce a fake $b$-tag.

  
  We attempt to $b$-tag only the four (two) hardest small-$R$ jets
  in the resolved (intermediate) category.
  %
  Attempting to $b$-tag all of the
  small-$R$ jets that satisfy the acceptance cuts worsens the
  overall performance as the probability of fake $b$-tags increases
  substantially.

  \item {\it Large-$R$ jets}.

    Large-$R$ jets are $b$-tagged by
    ghost-associating anti-$k_T$ $R=0.3$ (AKT03)
    subjets to the original large-$R$
    jets~\cite{Cacciari:2007fd,Aad:2013gja,
      ATLAS-CONF-2014-004,Aad:2015uka}.
    %
    A large-$R$ jet is considered $b$-tagged if both
    the leading and subleading AKT03 subjets, where the ordering
    is done in the subjet $p_T$, are both individually $b$-tagged,
    with the same criteria as the small-$R$ jets.
    %
     Therefore, a large-$R$ jet where the two leading
    subjets have at least one $b$-quark will be tagged
    with probability $f_b^2$.
    
    As in the case
    of small-$R$ jets, we only attempt to $b$-tag the two leading subjets,
    else one finds a degradation of the
    signal significance.
    %
    The treatment of the $b$-jet mis-identification
    from light and charm jets
    is the same as for the small-$R$ jets.
  
\end{itemize}

For the $b$-tagging probability $f_b$, along with
the $b$-mistag probability of light ($f_l$) and charm ($f_c$) jets,
we use the values $f_b=0.8$, $f_l=0.01$
and  $f_c=0.1$.


\subsection{Event categorisation}
\label{sec:categorisation}

The present analysis follows a strategy similar to the
scale-invariant resonance tagging of Ref.~\cite{Gouzevitch:2013qca}.
%
Rather than restricting to a specific event topology,
we aim to consistently combine the information from
the three possible topologies: boosted, intermediate and
resolved, with the optimal cuts for each category being determined
separately.
%
This approach is robust
under variations of
the underlying production model of Higgs pairs,
for instance in the case of
BSM dynamics, which can substantially increase
the degree of boost in the final state.


The three categories are defined as follows:
\begin{itemize}
\item {\it Boosted category}.

  An event which
  contains at least two large-$R$ jets, with the two leading jets
 being $b$-tagged.
 %
 Each of these two  $b$-tagged, large-$R$ jets are 
 therefore candidates
 to contain the decay products of a Higgs boson.

\item {\it Intermediate category}.

  An event with exactly one  $b$-tagged, large-$R$ jet, which
  is assigned to be the leading Higgs candidate.
  %
  In addition, we require at least two $b$-tagged, small-$R$ jets,
  which must be separated with respect to the large-$R$ jet
  by an angular distance of $\Delta R\ge 1.2$.
  %
    
  The subleading Higgs boson candidate is reconstructed
  by selecting the two $b$-tagged small-$R$ jets that minimize the difference
  between the invariant mass of the large-$R$ jet
  with that of the dijet obtained
  from the sum of the two small-$R$ jets.
  
\item {\it Resolved category}.

  
 An event with at least
  four $b$-tagged small-$R$ jets.
  %
  The two Higgs candidates are reconstructed out of the
  leading four small-$R$ jets in the event
  by considering all possible combinations of forming two pairs of jets
  and then choosing the configuration that minimizes the relative difference of
  dijet masses.
  %
  
\end{itemize}


Once a Higgs boson candidate has been identified,
its invariant mass is required to lie within a fixed window
of width $80~{\rm GeV}$ around the nominal Higgs boson mass of $m_h= 125$
GeV.
%
Specifically we require the condition
\be
\label{higgsmasswindow}
|m_{h,j} - 125| < 40~{\rm GeV} \, ,\, j=1,2 \, ,
\ee
where $m_{h,j}$ is the invariant mass of each of the two reconstructed  Higgs candidates.
%
This cut is substantially looser than the corresponding
cut used in the typical ATLAS and CMS $h\to b\bar{b}$
analyses~\cite{Aad:2012gxa,Chatrchyan:2013zna}.
%
The motivation
for such a loose cut 
is that further improvements of the
signal significance will be obtained using an MVA.
%
Only events where the two Higgs candidates satisfy
Eq.~(\ref{higgsmasswindow}) are classified as signal events.
%

These three categories are not exclusive:
a given event can be assigned to more than one category, for
example, satisfying the requirements of both the intermediate
and resolved
categories at the same time.
%
The exception is the boosted and intermediate categories, which have
conflicting jet selection requirements.
%

This is achieved as follows.
 %
First of all we perform an inclusive analysis, and optimise the
signal significance
$S/\sqrt{B}$ in each of the three categories separately, including
the MVA.
%
We find that the category with highest significance is
the boosted one,
followed by the intermediate and the resolved topologies, the latter two
with similar significance.
%
Therefore, when ascertaining which category an event is to be exclusively placed:
if the event satisfies the boosted requirements, it is assigned to
this category, else we check if it suits the intermediate
requirements.
%
It the event also fails the intermediate category
requirements, we
then check if it pass the resolved selection criteria.
%
The resulting exclusive event samples are then separately processed
through the MVA, allowing a consistent combination
of the significance of the three event categories.

\subsection{Motivation for basic kinematic cuts}

We now motivate the
kinematic cuts applied to the different categories, 
comparing representative kinematic distributions between
signal and background events.
%
Firstly we present results without PU, and then
discuss
the impact of PU
on the description of the kinematic
distributions.
%
In the following, all
distributions are normalized to their total integral.


In Fig.~\ref{fig:cutplots1} we show
the $p_T$ distributions
of the
  leading and subleading large-$R$ jets in the boosted category.
  %
  We observe that the background distribution
falls off more rapidly as a function of $p_T$ than the di-Higgs signal.
  %
  On the other hand, the cut in $p_T$ cannot be too strong to avoid
  a substantial degradation of signal selection efficiency,
  specially taking into account the subleading large-$R$ jet.
  %
  This comparison justifies the cut of $p_T \ge 200$ GeV
  for the large-$R$ jets that we impose in the boosted category.
  

%%%%%%%%%%%%%%%%%%%%%%%%%%%%%%%%%%%%
\begin{figure}[t]
\begin{center}
 \includegraphics[width=0.48\textwidth]{plots/pt_H0_bst_C1d_noPU.pdf}
 \includegraphics[width=0.48\textwidth]{plots/pt_H1_bst_C1d_noPU.pdf}
\caption{\small  Comparison of the $p_T$ distributions of the
  leading (left) and
  subleading (right) large-$R$ jets in the boosted category,
  for signal and background events.
  %
  Distributions have been normalized to unity.
  %
  The total background is the sum of all components
  listed in Table~\ref{tab:samples}.
}
\label{fig:cutplots1}
\end{center}
\end{figure}
%%%%%%%%%%%%%%%%%%%%%%%


Another selection requirement for the boosted category is that the two
leading AKT03 subjets of the large-$R$ jet
should satisfy $p_T \ge $ 50 GeV.
%
To motivate this cut, in Fig.~\ref{fig:cutplots22}
we show the distribution in $p_T$ of the leading
and subleading AKT03 subjets in the subleading large-$R$ jet in events
corresponding to the boosted category.
%
It is clear from the comparison that the subjet $p_T$ spectrum is
relatively harder in the signal with respect to the background.
%
On the other hand, considering the subleading AKT03 subjet,
this cut in $p_T$
cannot be too harsh, to maintain a high signal selection
efficiency.
%
Therefore,
as for the previous distribution, also here 
a compromise between suppressing backgrounds but keeping a large fraction of
signal events is crucial.


%%%%%%%%%%%%%%%%%%%%%%%%%%%%%%%%%%%%
\begin{figure}[t]
\begin{center}
 \includegraphics[width=0.48\textwidth]{plots/pt_leadSJ_fj2_noPU.pdf}
 \includegraphics[width=0.48\textwidth]{plots/pt_subleadSJ_fj2_noPU.pdf}
 \caption{\small  Same as Fig.~\ref{fig:cutplots1} for the leading (left)
   and subleading (right) AKT03
   subjets in the subleading Higgs candidate large-$R$ jet.
}
\label{fig:cutplots22}
\end{center}
\end{figure}
%%%%%%%%%%%%%%%%%%%%%%%


Turning to the resolved category, an important aspect to account for
in the  selection
cuts is the fact that the $p_T$ distribution
of the four leading small-$R$ jets of the event can be relatively soft,
specially for the subleading jets.
%
As noted in~\cite{deLima:2014dta}, this is due to the fact
that the boost from the Higgs decay is moderate,
therefore the $p_T$ selection cuts for the small-$R$ jets cannot be too large.
%
In Fig.~\ref{fig:cutplots23}
we show the distribution in $p_T$ of the four leading
small-$R$ jets in signal and background events: we observe that both
distributions peak at $p_T \le 50$ GeV, with the signal distribution
falling off less steeply at large $p_T$.
%
The feasibility of triggering on four small-$R$ jets with a relatively
soft $p_T$ distribution is one of the experimental challenges for
exploiting the resolved category in this final state,
and hence the requirement that $p_T \ge 40$ GeV for
the small-$R$ jets.
%
In  Fig.~\ref{fig:cutplots23} we also show the
rapidity distribution of the the small-$R$
jets in the resolved category.
%
As expected, the production
is mostly central, more so in the case
of signal events, since backgrounds are dominated by
QCD $t$-channel exchange, therefore the
selection criteria on the jet rapidity are very efficient.


%%%%%%%%%%%%%%%%%%%%%%%%%%%%%%%%%%%%
\begin{figure}[t]
\begin{center}
 \includegraphics[width=0.48\textwidth]{plots/pt_smallRjets_res_noPU.pdf}
 \includegraphics[width=0.48\textwidth]{plots/eta_smallRjets_res_noPU.pdf}
 \caption{\small Same as Fig.~\ref{fig:cutplots1}, now for the
   $p_T$ and rapidity distributions of the small-$R$
   jets corresponding to the resolved selection.
}
\label{fig:cutplots23}
\end{center}
\end{figure}
%%%%%%%%%%%%%%%%%%%%%%%

One of the most discriminating selection cuts is the requirement
that the invariant mass of the Higgs candidate (di)jets must lie within a window
around the nominal Higgs value, Eq.~(\ref{higgsmasswindow}).
%
In Fig.~\ref{fig:mHHinv} we show the invariant mass
of the leading reconstructed Higgs candidates, before the Higgs mass window
selection
  is applied, for the resolved and boosted categories.
%
While the signal distribution naturally peaks at the
nominal Higgs mass, the background distributions
show no particular
structure.
%
The
width of the Higgs mass peak is driven both from QCD effects,
such as initial-state radiation (ISR)
and out-of-cone radiation, as well
as from the four-momentum smearing applied to final state particles
as part of our minimal detector simulation.
%

%%%%%%%%%%%%%%%%%%%%%%%%%%%%
\begin{figure}[t]
\begin{center}
  \includegraphics[width=0.48\textwidth]{plots/m_H0_res_C1d_noPU.pdf}
  \includegraphics[width=0.48\textwidth]{plots/m_H0_bst_C1d_noPU.pdf}
  \caption{\small Same as   Fig.~\ref{fig:cutplots1} for the invariant
    mass distribution of the leading Higgs candidates in the resolved
    (left) and boosted (right) selections.
}
\label{fig:mHHinv}
\end{center}
\end{figure}
%%%%%%%%%%%%%%%%%%%%%%%

The invariant mass of the di-Higgs system is another important
kinematic distribution for this process.
%
The di-Higgs invariant mass is a direct measure of the boost of the system,
which in  BSM scenarios can be substantially
enhanced, for instance due to
specific $d=6$ EFT operators~\cite{Azatov:2015oxa}.
%
One important advantage of the $b\bar{b}b\bar{b}$ final state for
di-Higgs production is that it significantly increases the reach
in $m_{hh}$ as compared to other channels with smaller branching
ratios,
such as $2b2\gamma$ or $2b2\tau$.
%
In Fig.~\ref{fig:mhh} we show the invariant mass distribution of the
reconstructed Higgs pairs,
comparing the resolved and the boosted categories.

%%%%%%%%%%%%%%%%%%%%%%%%%%%%%%%%%%%%%%%%%%%%%%%
\begin{figure}[t]
\begin{center}
  \includegraphics[width=0.48\textwidth]{plots/m_HH_C2_res_noPU.pdf}
  \includegraphics[width=0.48\textwidth]{plots/m_HH_C2_bst_noPU.pdf}
  \caption{\small
Same as   Fig.~\ref{fig:cutplots1} for the invariant
mass distribution of the di-Higgs system $m_{hh}$, in
the resolved (left) and boosted (right) categories.
}
\label{fig:mhh}
\end{center}
\end{figure}
%%%%%%%%%%%%%%%%%%%%%%%%%%%%%%%%%%%%%%%%%%%%%%%%%

In the resolved case, we see that the distribution
in $m_{hh}$ is rather harder for the signal as compared
to the background,
and therefore one expects that cutting in $m_{hh}$ would help signal
discrimination.
%
For the boosted category the overall trend of the $m_{hh}$ distribution
is different because of the selection criteria, and the
distribution now peaks at higher values of the invariant mass.
%
In this case, signal and background distributions are not significantly
differentiated.
%
Note that at parton-level the $m_{hh}$ distribution
for signal events has a kinematic
cut-off at $m_{hh}^{\rm min}=250$ GeV, which is smeared due
to parton shower and detector resolution effects.

In Fig.~\ref{fig:pthh} we show the transverse momentum of
the di-Higgs system, $p_T^{hh}$,
for the resolved and boosted categories.
%
Once more we see that the background has a steeper fall-off in $p_T^{hh}$
than the signal, in both categories, therefore this variable
should provide additional discrimination power, motivating its inclusion
as one of the inputs for the MVA.
%
In our LO simulation the $p_T^{hh}$ distribution is generated
by the parton shower, an improved theoretical
description would require
merging higher-multiplicity
matrix elements~\cite{Maierhofer:2013sha} or matching to
the NLO calculation~\cite{Frederix:2014hta},
%

%%%%%%%%%%%%%%%%%%%%%%%%%%%%
\begin{figure}[t]
\begin{center}
  \includegraphics[width=0.48\textwidth]{plots/pt_HH_C2_res_noPU.pdf}
  \includegraphics[width=0.48\textwidth]{plots/pt_HH_C2_bst_noPU.pdf}
  \caption{\small Same as Fig.~\ref{fig:cutplots1} for the transverse momentum
    distribution of the di-Higgs system $p_T^{hh}$.
}
\label{fig:pthh}
\end{center}
\end{figure}
%%%%%%%%%%%%%%%%%%%%%%%

We shall now investigate the discrimination power
provided by jet substructure
quantities.
%
In Fig.~\ref{fig:mva_substructure_1}
we show the distributions of representative
 substructure variables for the boosted category: the
$k_t$ splitting scale $\sqrt{d_{12}}$, Eq.~(\ref{eq:ktsplitting}), 
the ECF ratio $C_2^{(\beta)}$,
Eq.~(\ref{eq:c2}), and
the 2--to--1 subjettiness ratio $\tau_{12}$, Eq.~(\ref{eq:tau21}),
all for the leading
Higgs candidates, and also $\tau_{12}$ for the subleading
Higgs candidates.

%%%%%%%%%%%%%%%%%%%%%%%%%%%%%%%%%%%%%%%%%%%%%%%%%%%%%%%
\begin{figure}[t]
  \begin{center}
    \includegraphics[width=0.48\textwidth]{plots/split12_h1_C1_boost.pdf} 
  \includegraphics[width=0.48\textwidth]{plots/EEC_C2_h0_C1_boost.pdf}
  \includegraphics[width=0.48\textwidth]{plots/tau21_h0_C1_boost.pdf}
  \includegraphics[width=0.48\textwidth]{plots/tau21_h1_C1_boost.pdf}
  \caption{\small Distribution of representative substructure variables
    in the boosted category at the end of the cut-based
    analysis, to be used as input to the MVA.
    %
    From top to bottom and from left to right we show  the
    $k_t$ splitting scale $\sqrt{d_{12}}$,
    the energy correlation ratio $C_2^{(\beta)}$
    and the  subjettiness ratio $\tau_{21}$ for the leading
    Higgs.
    %
    In the case of 
    $\tau_{21}$  the distributions for the subleading Higgs are also given.
 }
\label{fig:mva_substructure_1}
\end{center}
\end{figure}
%%%%%%%%%%%%%%%%%%%%%%%%%%%%%%%%%%%%%%%%%%%%%%%%%%%%%%%%%%%%%%%%%%%

From Fig.~\ref{fig:mva_substructure_1}
we observe how for these substructure variables the shapes of the signal
and background distributions reflect
the inherent differences in the internal structure of
QCD jets and jets originating from Higgs decays.
%
Signal and background distributions peak
in rather
different regions. For example, the $k_t$ splitting scale $\sqrt{d_{12}}$
peaks around 80 GeV (40 GeV) for signal (background) events, while
the distribution of the
ECF ratio $C_2^{(\beta)}$ is concentrated at small values
for signal and is much broader for background events.
%
From Fig.~\ref{fig:mva_substructure_1} we also see
the distributions of the subjettiness ratio $\tau_{12}$ are
reasonably similar
for both the leading and the subleading jets.
%

\subsection{Impact of pile-up}
\label{sec:pileup}

Now we turn to discuss how the description of kinematic
distributions for $hh\to 4b$ signal
and background processes are
modified in the presence of pile-up.
%
To study the impact of PU,
Minimum Bias events have been generated
with {\tt Pythia8}, and then
superimposed to the signal
and background samples described in Sect.~\ref{mcgeneration}.
%
We have explored two scenarios,
one with a mean number of
PU vertices per bunch crossing of $\la n_{\rm PU}\ra=80$,
and another
with $\la n_{\rm PU}\ra=150$.
%
While the latter is closer to the HL-LHC forecasts,
we adopt $\la n_{\rm PU}\ra=80$ as our baseline,
given that it
is beyond the scope of this paper to fully optimize
the PU subtraction.

%
In order to subtract PU in hadronic collisions, a number of techniques
are available~\cite{Cacciari:2009dp,TheATLAScollaboration:2013pia,Butterworth:2008iy,Cacciari:2007fd,Krohn:2009th,Krohn:2013lba,Ellis:2009me,Bertolini:2014bba,Cacciari:2014gra,Cacciari:2014jta,Berta:2014eza,Larkoski:2014wba}.\footnote{
These techniques have also important applications in the subtraction
of the UE/MPI contamination for jet reconstruction
in heavy ion collisions~\cite{Cacciari:2010te}.
}
%
In this work, PU  is subtracted
with the {\tt SoftKiller} (SK)
method~\cite{Cacciari:2014gra}, as implemented in {\tt FastJet},
whose performance has been shown to
improve the commonly used area-based subtraction~\cite{Cacciari:2009dp}.
%
The idea underlying {\tt SoftKiller} consists of eliminating particles
below a given cut-off in their transverse momentum, $p_T^{\rm (cut)}$, whose
value is dynamically determined in a way that makes the event-wide
transverse-momentum flow density $\rho$ vanish.
%
This $p_T$ flow density is defined as
\be
\rho\equiv{\rm median}_i \Bigg\{ \frac{p_{Ti}}{A_i}\Bigg\} \, ,
\ee
where the median is computed over all the regions $i$ with area
$A_i$ and transverse momentum $p_{Ti}$ in which the $\lp \eta,\phi\rp$ plane
is partitioned.
%
From its definition in terms of the median,
it follows that the value of $p_T^{(\rm cut)}$
will be dynamically raised until half of the regions have
$p_{Ti}=0$.
%
The size and number of these regions is a free parameter of the algorithm -
here we will use square regions with length $a=0.4$.
%
We restrict ourselves to the central rapidity region,
$|\eta| \le 2.5$, for the estimation of the
$p_T$ flow density $\rho$.
%
The {\tt SoftKiller} subtraction is applied to particles at the end of the parton shower, before
jet clustering.


%%%%%%%%%%%%%%%%%%%%%%%%
\begin{figure}[t]
  \begin{center}
  \includegraphics[width=0.63\textwidth]{plots/m_htot_res_signal_PUnoSK.pdf}
    \caption{\small
    The invariant mass distributions of Higgs candidates in signal
events in the resolved category, comparing the results without PU,
with PU $\la n_{\rm PU}\ra=80$. The distribution is then compared with and without SK subtraction.
}
\label{fig:PUvalidation}
\end{center}
\end{figure}
%%%%%%%%%%%%%%%%%%%%%%%

To validate the subtraction of PU,
we compare representative kinematic distributions
without PU, with PU, and with PU after SK subtraction.
%
In Fig.~\ref{fig:PUvalidation} we show the
invariant mass distributions of Higgs candidates for signal
events in the resolved category.
%
If PU is not subtracted there is a large shift in the Higgs mass
peak, by about 40 GeV.
%
Once SK subtraction is performed, we recover a distribution much closer
to the no PU case, with only a small shift of $\simeq 5$ GeV
and a broadening of the mass
distribution.
%

In Fig.~\ref{fig:m_H_PU} we show the invariant mass distribution
of the leading and subleading Higgs candidates,
this time corresponding
to the boosted category.
%
These distributions are plotted
after
$b$-tagging, that is,
before they are used as input to the MVA.
%
The residual effects of PU
after the {\tt SoftKiller} subtraction are smaller
in the boosted selection as compared to the resolved case,
with the position of the Higgs mass peak being essentially
unchanged, and only a moderate broadening of the
mass distribution.
%

%%%%%%%%%%%%%%%%%%%%%%%%
\begin{figure}[t]
  \begin{center}
      \includegraphics[width=0.49\textwidth]{plots/m_H0_bst_comp.pdf}
      \includegraphics[width=0.49\textwidth]{plots/m_H1_bst_comp.pdf}
  \caption{\small
    Comparison of the invariant mass distributions of the leading (left)
    and subleading (right) Higgs candidates
    in the boosted category,
    both without PU and with
    PU, $\la n_{PU}\ra=80$, subtracted by {\tt SoftKiller}.
}
\label{fig:m_H_PU}
\end{center}
\end{figure}
%%%%%%%%%%%%%%%%%%%%%%%

In Fig.~\ref{fig:mHH_PU}
we compare the transverse momentum of the leading Higgs
candidate, $p_t^{h_1}$ and the invariant mass of the di-Higgs system
$m_{hh}$, in both the boosted and resolved categories.
%
In the case of the $p_T^{h_1}$ distribution, the differences between the selection
criteria for the resolved
and boosted categories is reflected in the rightward shift of the latter.
%
The effect of PU is small in the boosted case, and generally moderate
in the resolved case, except for at large $p_T$.
%
A similar behaviour is observed in the di-Higgs invariant mass distribution.
%
Therefore we can consider the PU subtraction strategy
as validated for the purposes of this study, although
further optimisation should still be possible.

%%%%%%%%%%%%%%%%%%%%%%%%
\begin{figure}[t]
  \begin{center}
  \includegraphics[width=0.49\textwidth]{plots/pt_H0_C2_res_comp.pdf}
  \includegraphics[width=0.49\textwidth]{plots/pt_H0_C2_bst_comp.pdf}
  \includegraphics[width=0.49\textwidth]{plots/m_HH_C2_res_comp.pdf}
  \includegraphics[width=0.49\textwidth]{plots/m_HH_C2_bst_comp.pdf}
  \caption{\small
    Same as Fig.~\ref{fig:m_H_PU} for
   the transverse momentum $p_T^h$ of the leading
    Higgs candidate (upper plots) and of the invariant mass $m_{hh}$
    of the di-Higgs system (lower plots) in the resolved
    (left) and boosted (right) categories.
}
\label{fig:mHH_PU}
\end{center}
\end{figure}
%%%%%%%%%%%%%%%%%%%%%%%

We can also assess the impact of PU in some of
substructure variables that will be 
used as input to the MVA in the boosted category.
%
In Fig.~\ref{fig:Substructure_PU} we show the 2-to-1 subjettiness ratio
$\tau_{21}$, Eq.~(\ref{eq:tau21}), and the ratio
of energy correlation functions, $D_2^{(\beta)}$,
Eq.~(\ref{eq:d2}), for the leading Higgs candidate.
%
Comparing the signal distributions in the cases without PU and
with PU subtracted with SK, we observe that both substructure variables
are reasonably robust in a environment including PU.
%
The ratio of energy correlation functions $D_2^{(\beta)}$ 
was indeed explicitly constructed~\cite{Larkoski:2013eya}
to be robust against PU contamination.
%

%%%%%%%%%%%%%%%%%%%%%%%%
\begin{figure}[t]
  \begin{center}
  \includegraphics[width=0.49\textwidth]{plots/D2_h0_bst_comp.pdf}
  \includegraphics[width=0.49\textwidth]{plots/tau21_h0_bst_comp.pdf}
  \caption{\small
    Same as Fig.~\ref{fig:m_H_PU} for the
    substructure variables $D_2^{(\beta)}$ (left)
     and $\tau_{21}$ (right)
     for the leading Higgs candidates in the boosted category.
}
\label{fig:Substructure_PU}
\end{center}
\end{figure}
%%%%%%%%%%%%%%%%%%%%%%%

It is also interesting to quantify how
the relative differences between
signal over background distributions are modified by the inclusion of PU.
%
Considering first of all the boosted category,
in Fig.~\ref{fig:signal-vs-back-boosted} we compare
various kinematic distributions for signal and background events,
with and without PU for the leading Higgs candidate: the invariant mass, the transverse
momentum distribution $p_T$,
     the 2--to--1 subjettiness ratio $\tau_{21}$, and 
     the $k_T$ splitting scale $\sqrt{d_{12}}$.
     %
      We verify that the relevant
      qualitative differences between signal
      and background distributions are maintained  in the presence of PU.
      %
      This is especially noticeable for the substructure variables, which
      exhibit a similar discrimination power both with and without
      PU.
     

%%%%%%%%%%%%%%%%%%%%%%%%
\begin{figure}[t]
  \begin{center}
  \includegraphics[width=0.49\textwidth]{plots/pt_h0_bst_comp_back.pdf}
   \includegraphics[width=0.49\textwidth]{plots/tau21_h1_bst_comp_back.pdf}
   \includegraphics[width=0.49\textwidth]{plots/split12_h0_bst_comp_back.pdf}
   \includegraphics[width=0.49\textwidth]{plots/m_h0_bst_comp_back.pdf}
   \caption{\small
     Comparison of kinematic distributions, in
     the boosted category, for signal and background events
    in the case of PU subtracted with SK:
     the transverse momentum  $p_T$ and
     and the substructure variables $\tau_{21}$ and $\sqrt{d_{12}}$
     for the leading Higgs candidate, and the invariant
     mass of the di-Higgs system $m_{hh}$.
     %
 }
\label{fig:signal-vs-back-boosted}
\end{center}
\end{figure}
%%%%%%%%%%%%%%%%%%%%%%%



We can also perform a similar comparison for
the resolved category.
%
In Fig.~\ref{fig:signal-vs-back-resolved} we compare
the kinematic distributions for signal and background events,
     with and without PU, for the invariant mass and the $p_T$ of the leading
     Higgs candidate.
     %
     Also in this
     case the PU-subtracted background distributions appear reasonably close
     to their no PU counterparts.

%%%%%%%%%%%%%%%%%%%%%%%%
\begin{figure}[t]
  \begin{center}
      \includegraphics[width=0.49\textwidth]{plots/pt_h0_res_comp_back.pdf}
   \includegraphics[width=0.49\textwidth]{plots/m_h0_res_comp_back.pdf}
     \caption{\small
       Same as Fig.~\ref{fig:signal-vs-back-boosted} for the resolved category.
}
\label{fig:signal-vs-back-resolved}
\end{center}
\end{figure}
%%%%%%%%%%%%%%%%%%%%%%%

To complete this discussion, it is illustrative to determine
which is the mass resolution obtained for the
reconstructed Higgs candidates in various cases.
%
In Table~\ref{tab:massresolution}
we indicate the shift of the invariant mass peak as compared
to the nominal Higgs mass, $\delta m_h$, as well as the
corresponding width of the distribution, $\sigma_{m_h}$,
obtained from fitting a Gaussian to the mass distributions
of both the leading and subleading Higgs candidates in
the resolved category.
        %
        We show results for three cases: without PU, with PU $\la n_{\rm PU}\ra=80$
        but without subtraction, and the same with SK subtraction.
%
        We find a mass resolution of around 9 GeV in the case
        without PU, which worsens only slightly
        to around 11 GeV in the case of $\la n_{\rm PU}\ra=80$ PU
        with SK subtraction.
        %
        We also note that after subtraction, the peak of
        the invariant mass distributions of Higgs candidates
        coincides with the nominal values of $m_h$ within a few GeV.

    %%%%%%%%%%%%%%%%%%%%%%%%%%%%%%%%%%%%%%%%%%%
    \begin{table}[h]
      \centering
      \begin{tabular}{|c|c|c|c|}
        \hline
        \multicolumn{4}{|c|}{Resolved category}\\
        \hline
        \hline
        &   &   $\la m_h^{\rm reco}\ra-m_h$  &  $\sigma_{m_h}$  \\
              \hline
        \multirow{2}{*}{no PU}  & leading Higgs  &  -3.8 GeV   & $\lp 8.5\pm 0.2\rp$ GeV   \\
          & subleading Higgs   & -5.8 GeV  &  $\lp 9.1\pm 0.3\rp$ GeV \\
        \hline
          \multirow{2}{*}{$\la n_{\rm PU}\ra=80$}  & leading Higgs  &  +33 GeV   & $\lp 8.8\pm 1.5\rp$ GeV   \\
          & subleading Higgs   & +31 GeV  &  $\lp 11.7\pm 3.3\rp$ GeV \\
          \hline
            \multirow{2}{*}{$\la n_{\rm PU}\ra=80$+SK}  & leading Higgs  &  +3.9 GeV   & $\lp 10.7\pm 0.3\rp$ GeV   \\
          & subleading Higgs   & +2.1 GeV  &  $\lp 10.5\pm 0.3\rp$ GeV \\
            \hline
            \multicolumn{4}{c}{}\\
             \hline
        \multicolumn{4}{|c|}{Boosted category}\\
        \hline
        \hline
        &   &   $\la m_h^{\rm reco}\ra-m_h$  &  $\sigma_{m_h}$  \\
              \hline
        \multirow{2}{*}{no PU}  & leading Higgs  &  +2 GeV   & $\lp 8.2\pm 0.5\rp$ GeV   \\
          & subleading Higgs   & +1 GeV  &  $\lp 8.8\pm 0.5\rp$ GeV \\
        \hline
              \multirow{2}{*}{$\la n_{\rm PU}\ra=80$+SK}  & leading Higgs  &  --   & --   \\
          & subleading Higgs   & --  &  -- \\
        \hline
        \end{tabular}
      \caption{\label{tab:massresolution}
        Resolution of the invariant mass distribution of
        reconstructed Higgs candidates in the resolved (upper table)
        and boosted (categories) category.
        %
        We show three cases: without PU, with PU $\la n_{\rm PU}\ra=80$
        but without subtraction, and the same with SK subtraction.
        %
        We indicate the shift of the invariant
        mass peak $\la m_h^{\rm reco}\ra$ as compared
        to the nominal Higgs mass $m_h=125$ GeV, as well as the width
        of a Gaussian fitted to the mass distribution, $\sigma_{m_h}$,
        for both the leading and subleading Higgs candidates.
      }
      \end{table}

%%%%%%%%%%%%%%%%%%%%%%%%%%%%%%%%%%%%%%%%%%%%%%%%%%%%%%%%%%%%
%%%%%%%%%%%%%%%%%%%%%%%%%%%%%%%%%%%%%%%%%%%%%%%%%%%%%%%%%%%%
%%%%%%%%%%%%%%%%%%%%%%%%%%%%%%%%%%%%%%%%%%%%%%%%%%%%%%%%%%%%




%%%%%%%%%%%%%%%%%%%%%%%%%%%%%%%%%%%%%%%%%%%%%%%%%%%
%%%%%%%%%%%%%%%%%%%%%%%%%%%%%%%%%%%%%%%%%%%%%%%%%%%
%%%%%%%%%%%%%%%%%%%%%%%%%%%%%%%%%%%%%%%%%%%%%%%%%%%
%%%%%%%%%%%%%%%%%%%%%%%%%%%%%%%%%%%%%%%%%%%%%%%%%%%
