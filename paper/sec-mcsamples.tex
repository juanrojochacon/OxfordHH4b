

\section{Theoretical modeling of signal and background processes}
\label{mcgeneration}

First of all in this section we discuss the settings of the
Monte Carlo generation of the signal and background
events used in this work.
%
We also discuss how we study the effects of finite detector
resolution by means of an smearing of the final-state
particles four-momentum.

\subsection{Higgs pair production by gluon-fusion}
%%%%%%%%%%%%%%%%%%%%%%%%%%


%%%%%%%%%%%%%%%%%%%%%%%%%%%%
\begin{figure}[t]
\begin{center}
  \includegraphics[width=0.90\textwidth]{plots/hhFeyn.pdf}
  \caption{\small Representative Feynman diagrams
    for Higgs pair production in gluon fusion at
    leading order.
    %
    Only the quark triangle diagram (left) is sensitive to the Higgs trilinear coupling
    $\lambda$.
    %
    In the SM, the fermion loops are dominated by the top quark contribution.
}
\label{fig:hhFeyn}
\end{center}
\end{figure}
%%%%%%%%%%%%%%%%%%%%%%%

Higgs pair production is simulated at leading order using
{\tt MadGraph5\_aMC@NLO}~\cite{Alwall:2014hca}.
%
We use a tailored {\tt MadGraph5\_aMC@NLO} model~\cite{Maltoni:2014eza} that simulates
gluon-gluon-fusion Higgs boson pair production including the effects
of the
exact form factors for the top triangle and box loops at leading
order, the latter taken from from~\cite{Plehn:1996wb}.\footnote{We note that since recently it is possible to compute also
loop-induced processes in {\tt MadGraph5\_aMC@NLO} without the need of using specific
models~\cite{Hirschi:2015iia}.}
%
We adopt the NNPDF 3.0 $n_f = 4$ LO set~\cite{Ball:2014uwa} with
$\alpha_S(m_Z^2)=0.118$
as interfaced via {\tt LHAPDF6}~\cite{Buckley:2014ana}.
%
We use the default settings of the renormalization and factorization
scale of the model.
%
The Higgs boson parameters are also the same in the default model,
in particular we use $m_h=125$ GeV, consistent with the latest
measurements from ATLAS and CMS~\cite{Aad:2014aba,Khachatryan:2014jba}.
%
The value of the Higgs trilinear coupling $\lambda$ is set to its
Standard Model value.
%
The calculation is performed in the
$N_f$=4 scheme and thus
takes into account the finite mass of the $b$-quarks.


In Fig.~\ref{fig:hhFeyn} we show representative Feynman diagrams
    for Higgs pair production in gluon fusion at
    leading order.
    %
    The non-trivial interplay between the diagram with a heavy quark box
    and that of the triangle (that can lead to constructive or destructive interference)
    complicated the extraction of
    the trilinear coupling
    $\lambda$ from the measurement of the Higgs pair
    production cross-section.

The resummed NNLO+NNLL calculations for Higgs pair production have become available recently~\cite{deFlorian:2015moa},
leading to a moderate enhancement of the order of
few percent as compared to the fixed-order NNLO calculation.
%
Therefore, to achieve the correct higher-order value of the  integrated cross-section,
we rescale our signal sample using the
NNLO inclusive calculation from Ref.~\cite{deFlorian:2013jea} accounting the threshold resummed
NNLL effects, which corresponds to using a $K$-factor $\sigma_{\rm NNLO+NNLL}/\sigma_{\rm LO}=2.4$, as indicated
in Table~\ref{tab:samples}.


%
Parton level events are then showered with the {\tt Pythia8} Monte
Carlo~\cite{Sjostrand:2007gs,Sjostrand:2014zea}, in particular with {\tt v8.201}.
%
We use the default settings in the parton shower, in particular
we use the Monash 2013 tune~\cite{Skands:2014pea}, based on the NNPDF2.3LO PDF set~\cite{Ball:2012cx}.
%
We have switched off hadronisation, since this allows a more transparent modeling of
$b$-tagging.
%
A complete simulation of $b$-tagging at the hadron level is difficult to achieve without a full
detector simulator, which is beyond the scope of this paper.

\subsection{Backgrounds: QCD multijet and $t\bar{t}$ production}

Now we turn to discuss the Monte Carlo generation of the relevant background processes.
%
All background samples are generated at leading-order
with the {\tt SHERPA} event generator~\cite{Gleisberg:2008ta}, v2.1.1.
%
As in the case of the signal generation, the NNPDF 3.0 $n_f = 4$ LO set with strong coupling
$\alpha_S(m_Z^2)=0.118$ is used for all samples.
%
Factorisation and renormalisation scales are set as $\mu_F=\mu_R=H_T/2$ for all
the samples.

We have considered the most relevant background
processes that can lead to an event
being tagged as a $hh\to 4b$ candidate.
%
First of all we have QCD $4b$ multi-jet production of course, but
we also generate QCD $2b2j$ and $4j$ multi-jet samples,
that can lead to the event being tagged in the case of multiple light
jets being mistagged as $b$-quarks.
%
While the light jet mistag probability is small, we find that
in general the $2b2j$ and $4j$ backgrounds cannot be neglected because
of their large cross-sections and enhancement from combinatorics, that
increase their contribution as compared to a naive estimate.
%
In addition, we find that there is an important contribution from the radiation
of $b$-quarks off light partons during the parton shower, which enhances the contribution
of parton-level $2b2j$ and $4j$ events being tagged as signal events.
%
Neglecting the  $2b2j$ and $4j$ backgrounds is particularly
important for the resolved and intermediate categories, but less
so for the boosted one.



In addition to the QCD multijet, we also generate $t\bar{t}$ samples
in the fully hadronic final state, which lead to a $2b4j$ signature that can
contribute to the fake rates, and that has a similar topology that
the corresponding QCD sample.
%
We have used a value of the top quark mass of $m_t=173.2$ GeV.
%
Semileptonic decays of top quarks can be easily removed by requiring
a lepton veto.

The LO cross-section of the background samples has been rescaled so that the integrated
distributions reproduce known higher-order QCD results.
%
For the $4b$ and $2b2j$ samples, a NLO/LO $K$-factor has been determined
using {\tt MadGraph5\_aMC@NLO}~\cite{Alwall:2014hca}, which turns out to be 1.6 and 1.3
respectively.
%
For the $4j$ sample, we rescale it using the {\tt BLACKHAT}~\cite{Bern:2011ep}
results that indicates
a NLO/LO $K$-factor of 0.6.
%
Finally, the LO cross-section for $t\bar{t}$ production has been rescaled
to match the NNLO+NNLL calculation of Ref.~\cite{Czakon:2013goa}, which leads
to a $K$-factor of 1.4.
%
The $K$-factors that we use to rescale all the background samples have been collected in
Table~\ref{tab:samples}.


At the generation level the following basic cuts are applied to
background events.
%
Each final state particle in the hard process must have $p_T \ge 20$ GeV, and be located
in the central region with
$| \eta | \le 3.0$.
%
In addition, at the matrix element level
all final state particles must be separated by a minimum $\Delta R_{\mathrm{min}} =0.1$.
%
We have checked that the generator-level cuts are loose enough as compared to the actual
analysis cuts.
%


Total cross-sections and details of the samples generated are shown in Table~\ref{tab:samples}.
%
First of all we see that the $t\bar{t}$ and QCD $4b$ samples are of
the same order of magnitude.
%
The $bbjj$ cross-section is more than a factor 200 as compared to the
$4b$ result, so in principle it should be subleading taking into
account the mistag rate, but as we will show below,
this is not true due both to combinatorics and to $b$-quark radiation
during the parton shower.
%
 For top quark production, only the hadronic final state is generated.
 %
 We also provide in each case the corresponding inclusive $K$-factor
  that is applied in each case to correctly normalize the distribution to the known
  higher-order results.
 


%%%%%%%%%%%%%%%%%%%
\begin{table}[h]
  \small
\begin{center}
\begin{tabular}{|c|c|c|c|c|c|}
\hline
Process &  Generator & $N_{\mathrm{evt}}$ & $\sigma_{\mathrm{LO}}$ (pb)  & $K$-factor \\
\hline
\hline
$pp \to hh$ &  {\tt MadGraph5\_aMC@NLO} & 100K & $1.71\times10^{-2}$  &  2.4  (NNLO+NNLL~\cite{deFlorian:2013jea,deFlorian:2015moa}) \\
\hline
\hline
$pp \to b\bar{b}b\bar{b}$ &  {\tt SHERPA}v2.1.1 & 3M &$1.12 \times10^3$  & 1.6 (NLO~\cite{Alwall:2014hca}) \\
$pp \to b\bar{b}jj$ &  {\tt SHERPA}v2.1.1 & 3M & $2.66 \times 10^5$ & 1.3 (NLO~\cite{Alwall:2014hca}) \\
$pp \to jjjj$ &  {\tt SHERPA}v2.1.1 & 3M  & $9.71\times 10^6$ &  0.6 (NLO~\cite{Bern:2011ep})\\
$pp \to t\bar{t}\to b\bar{b}jjjj$ &  {\tt SHERPA}v2.1.1 & 3M & $2.51\times 10^3$   & 1.4 (NNLO+NNLL~\cite{Czakon:2013goa})\\
\hline
\end{tabular}
\caption{\small Summary of signal and background samples generated,
  together with the corresponding generator-level LO cross-sections.
  %
  For top quark production, only the hadronic final state is generated.
  %
We also provide in each case the corresponding inclusive $K$-factor
  that is applied in each case to correctly normalize the distribution to the known
  higher-order results. \label{tab:samples}
} 
\end{center}
\end{table}%
%%%%%%%%%%%%%%%%%%%%%%%%%%%%%%%%%%%%%%%%%%%

As a cross-check of the {\tt SHERPA}
background cross-sections reported in Table~\ref{tab:samples}, we have produced leading order
multi-jet samples
using the {\tt MadGraph5\_aMC@NLO} program, and compared with the results for the same processes reported in
Ref.~\cite{Alwall:2014hca}.
%
For comparison with the latter numbers, 
we require in all samples four anti-$k_T$ $R=0.5$ jets with $p_T \ge 80 $ GeV, and the leading jet must have $p_T \ge 100$ GeV, and
also that all jets must be within an acceptance of $|\eta| \le 2.5 $.
%
We find agreement, within the scale uncertainties, between the {\tt MadGraph5\_aMC@NLO} and {\tt SHERPA} calculations of the multi-jet
backgrounds, so we can be confident that the set-up that we will use in this analysis is robust enough.


\subsection{Mimicking detector resolution}


While it is beyond the scope of this work to perform a full
detector simulation, it is still important to include an estimate of detector
effects in the analysis, in particular for the finite resolution
in energy and momentum, which will affect some kinematical variables, in particular
the invariant mass of the Higgs candidates.
%
For example, in the absence of any four-momentum smearing, the impact of the $m_h$
distributions of the Higgs candidates in the MVA
would be unrealistically large.


In this work we simulate the finite momentum resolution of the ATLAS and CMS
hadronic calorimeters by applying a Gaussian smearing
with mean zero and standard deviation $\sigma_E$.
%
In particular, we rescale the four-momentum of all
final-state particles, before jet clustering, using
a common factor for the energy $E$ and the length of the
three-momentum $\vec{p}$, while keeping the direction unchanged:
%
\be
\label{eq:smearing}
E_i \, \to \, E'_i= \lp 1+ r_i\cdot\sigma_E \rp\, E_i \, ,\quad
\vec{p}_i \, \to \, \vec{p}'_i= \lp 1+ r_i\cdot\sigma_E \rp\, \vec{p}_i \, , \quad
i=1,\ldots,N_{\rm part} \, ,
\ee
with $r_i$ an univariate Gaussian random number which is different for each
of the $N_{\rm part}$ particles in the event.
%
This smearing reflects the fact that the angular resolution of the LHC detectors
is better than the energy resolution.
%
We take as baseline value a momentum smearing
factor of $\sigma_E=5\%$.

It is also beyond the scope of this paper to
consider
the effects in our analysis of  Underlying Event (UE),
Multiple Parton Interactions (MPI) and in
particular
pile-up (PU).
%
While PU will be be one of the major problems at the HL-LHC, fortunately
a number of powerful methods for the subtraction of  UE, MPI and PU
effects 
have been developed recently~\cite{Cacciari:2009dp,TheATLAScollaboration:2013pia,Butterworth:2008iy,Cacciari:2007fd,Krohn:2009th,Krohn:2013lba,Ellis:2009me,Bertolini:2014bba,Cacciari:2014gra,Cacciari:2014jta,Berta:2014eza,Larkoski:2014wba}.
%
For these reasons, in our Monte Carlo analysis we switch off MPI and PU,
and postpone to a future work a full detector simulation
to quantify the robustness
of our analysis strategy with respect to the inclusion
(and its subtraction) of the large PU
expected at the HL-LHC.


We note that while substructure variables can be affected by PU, event after its
subtraction, our analysis should be relatively robust since we do not cut
over any specific variables, but input the full distributions to the MVA.
%
In particular, if after PU suppression a given variable is modified
but in the same way for signal and background samples, it should not
affect MVA discrimination.
