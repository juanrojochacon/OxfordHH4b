\section{Monte Carlo samples}
\label{mcgeneration}

\subsection{Signal}
%%%%%%%%%%%%%%%%%%%%%%%%%%

\subsection{Background}
All background samples are generated with the {\tt SHERPA} event generator, version 2.1.1. For the explicit runcards used in the generation, see Appendix~\ref{app:runcards}.

The NNPDF 3.0 $n_f = 4$ LO set with strong coupling $\alpha_S=0.118$ is used for all samples. At the generator level the following basic cuts are applied. Each final state particle in the hard process must have $p_T \ge 20$ GeV, and be located within $| \eta | \le 3.0$. All final state particles must be separated by a minimum $\Delta R_{\mathrm{min}} =0.1$. Factorisation and renormalisation scales are set as $\mu_F=\mu_R=H_T/2$. Total cross-sections and details of the samples generated are shown in Table~\ref{tab:samples}. 


%%%%%%%%%%%%%%%%%%%
\begin{table}[h]
\begin{center}
\begin{tabular}{|c|c|c|c|c|}
\hline
Process &  Generator & $N_{\mathrm{evt}}$ & $\sigma_{\mathrm{tot}}$ \\
\hline
\hline
$pp \to HH$ &  MG5\_aMC@NLO & 100K & $1.729\times10^{-2}$ pb \\
\hline
$pp \to b\bar{b}b\bar{b}$ &  SHERPA 2.1.1 & 3M &$1.121 \times10^3$ pb \\
$pp \to b\bar{b}jj$ &  SHERPA 2.1.1 & 3M & $2.659 \times 10^5$ pb \\
$pp \to jjjj$ &  SHERPA 2.1.1 & 3M  & $9.709\times 10^6$ pb \\
$pp \to t\bar{t}$ &  SHERPA 2.1.1 & 3M & $2.514\times 10^3$ pb \\
\hline
\end{tabular}
\caption{Summary of generated samples to date. All {\tt SHERPA} samples have a MC error of $0.05\%$.} \label{tab:samples}
\end{center}
\end{table}%
%%%%%%%%%%%%%%%%%%%%%%%%%%%%%%%%%%%%%%%%%%%

Although suffering from a large theory uncertainty, we can compare the result of our background samples against those presented in the MG5\_aMC@NLO paper~\cite{Alwall:2014hca}.
Here for comparison we require in all samples four anti-$k_T$ $R=0.5$ jets with $p_T \ge 80 $ GeV, and the leading jet must have $p_T \ge 100$ GeV. All jets must be within an acceptance of $|\eta| \le 2.5 $. In the case of the samples with $b$ quarks in the final state, these requirements are extended to the appropriate number of $b$-jets. For example, in the 2$b$2$j$ sample there must be at least two $b$-jets that pass the cuts outlined above.

In Table~\ref{tab:xsecs} this comparison is summarised for the $2b2j$ and $4b$ samples. Considering the large theory errors, agreement is reasonable in both instances.

\begin{table}[h]
\begin{center}
\begin{tabular}{|c|c|c|}
\hline
Process & $\sigma$ aMC@NLO & $\sigma$ Oxford (SHERPA) \\
\hline\hline
$b\bar{b}b\bar{b}$ & $5.050 \times  10^{-1}$ pb & $4.123\times10^{-1}$ pb \\ 
$b\bar{b}jj$ & $1.852 \times 10^2$ pb &$4.239 \times 10^2$ pb \\ 
$jjjj$ & -  & $4.450\times 10^4$ pb \\
\hline
\end{tabular}
\caption{Comparison of LO Oxford SHERPA cross-sections with those of the aMC@NLO paper. The aMC@NLO cross-sections come with a quoted $~50\%$ theory uncertainty.} \label{tab:xsecs}
\end{center}
\end{table}%
