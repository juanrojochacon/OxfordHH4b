%%%%%%%%%%%%%%%%%%%%%%%%%%%%%%%%%%%%%%%%%%

\section{Results of the cut-based analysis}

\label{sec:results}

In this section we present the results of the 
cut-based analysis and provide a
cut-flow for the different analysis steps.
%
We
compare our results with other recent studies,
and assess how the signal significance
is affected if only the $4b$ component of the
QCD multi-jet background is taken into account but not 
the $2b2j$ and $4j$ components.
%
We present now the results in the no PU scenario, the impact
of PU will be discussed in the next section.


\subsection{Cut-flow and signal significance}

We now compare the cross-sections for
signal and background events at various
stages of the analysis.
%
We consider all relevant backgrounds (see Sect.~\ref{mcgeneration}),
and discuss how results are modified if only the $4b$
component is kept.
%
At each stage of the cut-flow, we also provide the
the signal significance, $S/\sqrt{B}$, and the signal
over background ratio, $S/B$, corresponding to the
 HL-LHC with
an integrated luminosity of $\mathcal{L}
=3000$ fb$^{-1}$.


In Table~\ref{tab:cutflowdetails}
we summarize the definition of the different
levels of the cut-flow of the present analysis,
separated into the boosted, intermediate,
    and resolved categories.
    %
    From top to bottom, additional cuts are required in addition
    to the previous ones.
    %
    In more details, these different analysis levels are the following:
   

%%%%%%%%%%%%%%%%%%%%%%%%%%%%%%%%%%%%%%%%%%%%%%%%%%%%%%%%%%%%%%%%%%%%%%%%%%%%%%%%%%%%%%%%%
%%%%%%%%%%%%%%%%%%%%%%%%%%%%%%%%%%%%%%%%%%%%%%%%%%%%%%%%%%%%%%%%%%%%%%%%%%%%%%%%%%%%%%%%%
\begin{table}[t]
  \centering
  \begin{tabular}{|c|c|c|c|}
\hline
&  Boosted  &   Intermediate &  Resolved  \\
\hline
\hline
{\bf C0} &  \multicolumn{3}{c|}{Generator level} \\
\hline
{\bf C1a} & $N_{\rm jets}^{R10}\ge 2$ & $N_{\rm jets}^{R04}\ge 2$, $N_{\rm jets}^{R10}=1$  &
$N_{\rm jets}^{R04}\ge 4$ \\
\hline
{\bf  C1b} & \multicolumn{3}{c|}{+$p_T$ cuts} \\
{\bf C1c} & \multicolumn{3}{c|}{+rapidity cuts}\\
\hline
 {\bf C1d} & +$N_{\rm MDT}\ge 2$ & +$N_{\rm jets}^{R10}=1$ with MDT  &
 +Higgs reconstruction \\
 \hline
{\bf C1e} & \multicolumn{3}{c|}{ +$m_H$ window cut} \\
\hline
{\bf C2} & \multicolumn{3}{c|}{+$b$-tagging}    \\
\hline
  \end{tabular}
  \caption{\small Definition of the cut-flow levels for the boosted, intermediate
    and resolved selections.
    %
      \label{tab:cutflowdetails}
  }
\end{table}
%%%%%%%%%%%%%%%%%%%%%%%%%%%%%%%%%%%%%%%%%%%%%%%%%%%%%%%%%%%%%%%%%%%%%%%%%%%%%%%%%%%%%%%%%
%%%%%%%%%%%%%%%%%%%%%%%%%%%%%%%%%%%%%%%%%%%%%%%%%%%%%%%%%%%%%%%%%%%%%%%%%%%%%%%%%%%%%%%%%


    \begin{itemize}
    \item {\bf C0}:  generator level cross-sections, with
      mild generation cuts in the case of the background processes, and no
      generation cuts
      for the signal events.
      %
    \item {\bf C1a}:  check that we have at least
      two large-$R$ jets (in the boosted case),
      one large-$R$ jet and at least 2 small-$R$ jets (in the intermediate
      case) and at least four small-$R$ jets (in the resolved case).
      %
      No acceptance requirements are imposed.
    \item {\bf C1b}: require that the above jets should
      satisfy the corresponding $p_T$ thresholds,
      $p_T \ge 200$ GeV for large-$R$ jets and
      $p_T \ge 50$ GeV for small-$R$ jets.
    \item {\bf C1c}: same for the
      rapidity acceptance requirement.
    \item {\bf C1d}: the two leading large-$R$ jets must
      be mass-drop tagged in the boosted category.
      %
      In the intermediate category, the large-$R$ jet must also be mass-drop tagged.
      %
    \item {\bf C1e}: after the two Higgs candidates  have been reconstructed,
      their invariant masses are required to lie a window around $m_H$,
      in particular between 85 and 165 GeV.
          \item {\bf C2}: the
            $b$-tagging conditions are
            imposed, see
            Sect.~\ref{sec:btagging}.
      \end{itemize}
    Signal and background events satisfying all the analysis cut up to the
    C2 level
    are then used as input for the MVA training, to be described next
    in Sect.~\ref{sec:mva}.
    %
    In Table~\ref{tab:cutflow_noPU_1} we collect
    the values for cross-sections for signal and background processes
    for different levels of the cut flow.
    %
    Results are divided into the resolved, intermediate and boosted categories,
    and are inclusive up to the C2 level, where exclusivity is imposed.
    %

In Table~\ref{tab:cutflow_noPU_1} we also  provide the signal over
      background ratio, $S/B$, and the signal
      significance, $S/\sqrt{B}$, corresponding to an integrated
      luminosity of $\mathcal{L}=3$ ab$^{-1}$.
      %
      This is done both taking into account all the background components and only
      the $4b$ QCD background.
        %
      We find that after $b$-tagging, the  $2b2j$ component is
      of the same order of magnitude as the $4b$ component.
      %
      This implies that the signal significance at the end of the cut-based
      analysis is degraded by about a factor two due to the contribution
      of light and charm jets being mis-identified as $b$-jets.
    

%%%%%%%%%%%%%%%%%%%%%%%%%%%%%%%%%%%%%%%%%%%%%%%%%
\begin{table}[t]
  \centering
  \scriptsize
  \begin{tabular}{|l|cc|cccc|cc|cc|}
  \hline
\multicolumn{11}{|c|}{HL-LHC, Resolved category, no PU}\\
\hline
&  \multicolumn{6}{c|}{Cross-section [fb]} &  \multicolumn{2}{c|}{$S/B$}  &  \multicolumn{2}{c|}{$S/\sqrt{B}$}  \\
   &  $hh4b$ &  total bkg  &   $4b$    &  $2b2j$   &   $4j$    &
$t\bar{t}$ &
tot & $4b$ & tot & $4b$ \\
  \hline
  \hline
 C1a   & $9$  &   $2.2\cdot 10^8$   & $6.9\cdot 10^4$ & $1.5\cdot 10^7$ & $2.0\cdot 10^8$ & $2.1\cdot 10^5$ &  $4.0\cdot 10^{-8}$  &  $1.3\cdot 10^{-4}$  &  0.03   & 1.9         \\
 C1b   & $9$  &   $2.2\cdot 10^8$   & $6.9\cdot 10^4$ & $1.5\cdot 10^7$ & $2.0\cdot 10^8$ & $2.1\cdot 10^5$ &  $4.0\cdot 10^{-8}$  & $1.3\cdot 10^{-4}$   &  0.03   & 1.9         \\
 C1c   & $2.6$  &   $4.4\cdot 10^7$   & $1.6\cdot 10^4$ & $3.2\cdot 10^6$ & $4.1\cdot 10^7$ & $8.8\cdot 10^4$ &   $6.1\cdot 10^{-8}$  &  $1.6\cdot 10^{-4}$   &   0.02   & 1.1         \\
 C2    & $0.5$  &   $4.9\cdot 10^3$   & $1.7\cdot 10^3$ & $2.9\cdot 10^3$ & $2.1\cdot 10^2$ & $47$ &            $ 1.1\cdot 10^{-4}$   & $2.9\cdot 10^{-4}$   &   0.4   & 0.6       \\
\hline
\end{tabular}

  $\,$ \\
  \vspace{0.5cm}
  \begin{tabular}{|l|cc|cccc|cc|cc|}
  \hline
\multicolumn{11}{|c|}{HL-LHC, Intermediate category, no PU}\\
\hline
&  \multicolumn{6}{c|}{Cross-section [fb]} &  \multicolumn{2}{c|}{$S/B$}  &  \multicolumn{2}{c|}{$S/\sqrt{B}$}  \\
   &  $hh4b$ &  total bkg  &   $4b$    &  $2b2j$   &   $4j$    &
$t\bar{t}$ &
tot & $4b$ & tot & $4b$ \\
  \hline
  \hline
C1a     & 2.8  &   $8.4\cdot 10^7$   & $2.1\cdot 10^4$ & $5.3\cdot 10^6$ & $7.9\cdot 10^7$ & $3.3\cdot 10^4$ &  $3.4\cdot 10^{-8}$   & $1.3\cdot 10^{-4}$  &   0.02   & 1.1 \\
 C1b     & 2.6  &   $5.8\cdot 10^7$   & $1.4\cdot 10^4$ & $3.6\cdot 10^6$ & $5.5\cdot 10^7$ & $3.0\cdot 10^4$ &  $4.5\cdot 10^{-8}$   & $1.9\cdot 10^{-4}$  &   0.02  & 1.2\\
 C1c     & 0.5  &   $3.5\cdot 10^6$   & $8.7\cdot 10^2$ & $2.1\cdot 10^5$ & $4.3\cdot 10^7$ & $8.8\cdot 10^3$ &  $1.6\cdot 10^{-7}$   & $6.1\cdot 10^{-4}$  &   0.02   & 1.0\\
 C2      & 0.09  &  $1.8\cdot 10^2$   & $56$ & $96$ & $22$ & 3.1             & $5.3\cdot 10^{-4}$    & $1.6\cdot 10^{-3}$  &   0.4   & 0.6 \\
\hline
\end{tabular}

  $\,$ \\
  \vspace{0.5cm}
    \begin{tabular}{|l|cc|cccc|cc|cc|}
  \hline
\multicolumn{11}{|c|}{HL-LHC, Boosted category, no PU}\\
\hline
&  \multicolumn{6}{c|}{Cross-section [fb]} &  \multicolumn{2}{c|}{$S/B$}  &  \multicolumn{2}{c|}{$S/\sqrt{B}$}  \\
   &  $hh4b$ &  total bkg  &   $4b$    &  $2b2j$   &   $4j$    &
$t\bar{t}$ &
tot & $4b$ & tot & $4b$ \\
  \hline
  \hline
 C1a     & 3.9  &   $4.6\cdot 10^7$   & $1.1\cdot 10^4$ & $2.9\cdot 10^6$ & $4.3\cdot 10^7$ & $2.4\cdot 10^4$   &   $8.2\cdot 10^{-8}$   & $3.4\cdot 10^{-4}$ &  0.03   & 2.0   \\
 C1b     & 2.7  &   $3.7\cdot 10^7$   & $7.5\cdot 10^3$ & $2.1\cdot 10^6$ & $3.5\cdot 10^7$ & $2.2\cdot 10^4$   &   $7.4\cdot 10^{-8}$   & $3.7\cdot 10^{-4}$ &  0.03   & 1.7   \\
 C1c     & 1.0  &   $3.9\cdot 10^6$   & $8.0\cdot 10^2$ & $2.3\cdot 10^5$ & $3.7\cdot 10^6$ & $7.1\cdot 10^3$   &   $2.6\cdot 10^{-7}$   & $1.3\cdot 10^{-3}$ &  0.03   & 2.0   \\
 C2      & 0.16  &   $2.5\cdot 10^2$  & $53$ & $1.9\cdot 10^2$ & $13$ & 1.6               &   $5.7\cdot 10^{-4}$   & $2.7\cdot 10^{-3}$ &   0.5                 & 1.1   \\
\hline
\end{tabular}
%%%%%%%%%%%%%%%%%%%%%%%%%%%%%%%%%%%%%%%%%%%%%%%%%%%%%%%%%%%%%%%

    \caption{\small The cross-sections, in femtobarns,
      for the signal and the various background
      processes at different steps of the
      cut-flow, for the resolved (upper table),
      intermediate (middle table) and boosted
      (lower table) categories, for the analysis
      without PU.
      %
      In each case, we also provide the signal over
      background ratio, $S/B$, and the signal
      significance, $S/\sqrt{B}$, considering either
      the total background or only the $4b$ component, for
      a total integrated luminosity of $\mathcal{L}=3$ ab$^{-1}$.
      %
      The different levels of the cut-flow are summarized
      in Table~\ref{tab:cutflowdetails}.
 \label{tab:cutflow_noPU_1}}
\end{table}
%%%%%%%%%%%%%%%%%%%%%%%%%%%%%%%%%%%%



%
In the boosted category, at the end of the cut-based
analysis, we find that around $10^3$ events
are expected
at the HL-LHC, with however a very large number,
$\simeq 10^6$, of QCD background events.
%
This leads to a pre-MVA signal significance of
$S/\sqrt{B}=1.2$ and a signal over background
ratio of $S/B=1.4\%$.
%
We emphasize that such signal
significance could have been enhanced
by applying tighter selection requirements,
but this is not needed here thanks to the subsequent
optimisation  performed with the MVA.
%
From  Table~\ref{tab:cutflow_noPU_1}
it is also possible to compute the corresponding pre-MVA
expectations for the LHC Run II with
$\mathcal{L}=300$ fb$^{-1}$: one expects around
120 signal events, with signal significance dropping down to
$S/\sqrt{B}\sim 0.3$.
%


Both the intermediate and resolved categories benefit from higher signal yields,
specially in the resolved category, but this enhancement is compensated by the
corresponding
increase in the QCD multi-jet background.
%
In both categories
the signal significance is similar to that of the boosted category,
$S\sqrt{B}\simeq 0.9$ in both categories.
%
An important additional
drawback of the resolved case is
that $S/B$
is one order to magnitude smaller.
%
Combining the pre-MVA results
from the
of the boosted, intermediate and resolved categories,
we obtain an overall
significance for the observation of the Higgs pair production
in the $b\bar{b}b\bar{b}$ final
state at the HL-LHC with $\mathcal{L}=3$ ab$^{-1}$
of  $S/\sqrt{B} \simeq 1.8$.

\subsection{The role of light  and charm jet mis-identification}

One of the main differences of the present study as compared
to previous works is the inclusion of both irreducible
and reducible background components, which allows
quantifying
the impact of the modeling of light and charm jet mistags. 
%
Two recent studies that have also studied the
feasibility of SM Higgs pair production in the $b\bar{b}b\bar{b}$
final state are from the UCL group~\cite{Wardrope:2014kya} and from
the
Durham group~\cite{deLima:2014dta}.
%
The UCL study is based
on requiring at least four $b$-tagged $R=0.4$ anti-$k_T$ jets
in central acceptance with $p_T \ge 40$ GeV, which are
then used to construct dijets (Higgs candidates) with
$p_T \ge 150$ GeV, $85 \le m_{\rm dijet} \le 140$ GeV
and $\Delta R \le 1.5$ between the two components
of the dijet.
%
In addition to the basic selection cuts, the constraints
from additional kinematic variables are included by means of a 
Boosted Decision Tree (BDT) discriminant.
%
The backgrounds included are the $4b$ and
$2b2c$ QCD multijets, as well as
$t\bar{t}$, $Zh$, $t\bar{t}h$ and $hb\bar{b}$,
and a signal significance of $S/\sqrt{B}\simeq 2.1$ for the HL-LHC
is obtained.

The Durham group study~\cite{deLima:2014dta} requires events
to have two $R=1.2$ C/A jets with $p_T\ge 200$ GeV, with
two $b$-tagged subjets inside each large-$R$ jet with
$p_T \ge$ 40 GeV each.
%
To improve the separation between
signal and background, both the BDRS
method and the Shower Deconstruction (SD)~\cite{Soper:2011cr,Soper:2012pb}
technique are used.
%
The backgrounds considered are QCD $4b$ as well as $Zb\bar{b}$, $hZ$ and
$hW$.
%
At the HL-LHC, their best result is obtained by requiring two
SD-tagged large-$R$ jets, which leads to $S/\sqrt{B}\simeq 2.1$, with
 slightly inferior performance with the BDRS tagger.
 %
 
 From the results of Table~\ref{tab:cutflow_noPU_1}, we observe
 that the signal significance for the boosted, intermediate,
 and resolved categories is increased to 2.7, 1.6 and 1.6, respectively,
 when only the QCD $4b$ background is included.
 %
 Combining in this
 case the signal significance in the three categories,
 we
 obtain $S/\sqrt{B_{\rm 4b}}\simeq 3.5$, which is twice
 as large as the result found when
 all background components are included.
 %
 At this level, part of the improvement that we found as compared
 to~\cite{deLima:2014dta,Wardrope:2014kya} arises
 from the combination of the three exclusive event topologies,
 as opposed the exploitation of a single specific category.
 

 It is interesting to compare in each category the interplay
 between the reducible and irreducible components of the
 QCD backgrounds.
 %
 In all cases, the $4b$ and $2b2j$ have comparable
 magnitudes, taking into account the size that the higher-order QCD
 effects could induce in
 the corresponding normalizations.
 %
 On the other hand, the $4j$ component
is always smaller by at least
 one order of magnitude.
 %
  So while the $4j$ component can be safely
 neglected, the inclusion of the
 $2b2j$ component is essential to robustly assess the feasibility
 of measuring Higgs pairs in this final state, and this is markedly
 important
 in the boosted category.
 %
 This has the important
 consequence that a promising avenue to improve the prospects
 of this measurement would be to reduce, as much as possible,
 the light and charm jet mis-identification rate.


 In Fig.~\ref{fig:histoBack} we show a
 comparison
    of the shapes of the $4b$ and $2b2j$
    components of the QCD background for the transverse momentum
    $p_T$ of the leading
Higgs candidate in the resolved
and boosted  categories, and the same
 for the invariant
mass $m_{hh}$ of the
    reconstructed di-Higgs system.
    %
    The two components lead to a rather similar shape
    for the two distributions, with some interesting
    differences, for instance, in the boosted
    category the $4b$ component is harder for the
    $p_T^h$ distribution.
  %
    The $2b2j$ component leads to a longer
    tail at small values of the $m_{hh}$ invariant
    mass distribution.
%
    We also observe that the $2b2j$ distributions
    are affected by somewhat larger
    Monte Carlo fluctuations as compared to $4b$, despite the large size
of the initial sample.
%
These fluctuations illustrate the challenge
of an accurate modelling of the effects
of $b$-jet mis-identification in QCD processes, which requires
very large MC samples.

%%%%%%%%%%%%%%%%%%%%%%%%%%%%
\begin{figure}[t]
\begin{center}
 \includegraphics[width=0.49\textwidth]{plots/pt_h0_C2_res_back_noPU.pdf}
 \includegraphics[width=0.49\textwidth]{plots/pt_h0_C2_bst_back_noPU.pdf}
  \includegraphics[width=0.49\textwidth]{plots/m_hh_C2_res_back_noPU.pdf}
  \includegraphics[width=0.49\textwidth]{plots/m_hh_C2_bst_back_noPU.pdf}
  \caption{\small
    Upper plots: comparison
    of the shapes of the $4b$ and $2b2j$
components of the QCD background for the $p_T$ of the leading
Higgs candidate in the resolved
(left plot) and boosted (right plot) categories.
    %
Lower plots:  same comparison for the invariant
mass $m_{hh}$ of the
    reconstructed di-Higgs system.
}
\label{fig:histoBack}
\end{center}
\end{figure}
%%%%%%%%%%%%%%%%%%%%%%%

The reason why the  $2b2j$ process
cannot be neglected as compared to the $4b$ component
is the following.
%
In the boosted category, for example,
the cross-section  before
$b$-tagging is more than two orders
of magnitude larger in the $2b2j$
sample as compared to the $4b$ sample.
%
Now, after $b$-tagging, the naive expectation would
be a suppression of the former by a factor $(f_l/f_b)^2 \simeq
1.5\cdot 10^{-4}$, as compared to the $4b$ component,
since a total of four $b$-tags are required to classify the
event as a Higgs candidate.
%
So the ratio of $2b2j$ over $4b$ should be
around $\simeq 3\%$, and therefore negligible.
  %
While  we have checked that this expectation is borne
out at the parton level,
we find that  when parton shower effects
are accounted for, due both to radiation of $b\bar{b}$ pairs
and from selection effects, the situation is different.
%
In this case,
the
number of  $b$ quarks in the  final state is
increased substantially in the $2b2j$ component as compared
to the parton level content, while at the same
time the number of events in the $4b$ sample
with 4 $b$-jets is reduced.


We can make these statements more quantitative in the following way.
%
To first approximation, and neglecting the contribution from
charm mis-identification,
the
overall efficiency of the $b$-tagging requirements in the resolved category will be
given by the following expression:
\be
\label{btaggingeff}
{\rm EFF}_{\rm b-tag}\simeq \sum_{j=0}^{4}n^{\rm (b-jet)}_j\cdot f_b^{j}\cdot f_l^{4-j} \, ,
\ee
with $n^{\rm (b-jet)}_j$ the fraction of events, satisfying all the selection cuts,
where $j$ jets, out of the leading four jets of the event,
contain $b$ quarks (with $p_T^b\ge 15$
GeV).
%
Similar expressions can be derived to the boosted and intermediate categories.
%
The naive expectation is that all events in the $4b$ sample have $n^{\rm (b-jet)}_4\simeq 1$
and $n^{\rm (b-jet)}_j\simeq 0$ for $j\ne 4$, while the events in the $2b2j$ sample
should have $n^{\rm (b-jet)}_2\simeq 1$ and zero otherwise.
%
This leads to a ratio of overall $b$-tagging selection efficiencies
\be
\label{eq:naive}
\frac{ {\rm EFF}_{\rm b-tag} \lc 2b2j \rc}{{\rm EFF}_{\rm b-tag} \lc 4b\rc}
  \simeq
 \lp \frac{f_l}{f_b}\rp^2 \simeq 1.5\cdot 10^{-4} \, .
\ee
However, after the parton shower the above is not a good approximation.
%
First of all, we will have a non-negligible fractions $n^{\rm (b-jet)}_j$
with $j=3,4$ also in the $2b2j$ sample, due to $b$-quark pair radiation
during the shower.
%
Secondly, not all events in the $4b$ sample will lead to four small-$R$ $b$-jets,
due to a combination of selection cuts and
parton shower effects.
%

%%%%%%%%%%%%%%%%%%%%%%%%%%%%%%%%%%%%%%%%%%%%%%%%
\begin{table}[t]
  \centering
  \small
  \begin{tabular}{|c|c|c|c|c|c|c|c|}
    \hline
  \multicolumn{2}{|c|}{}   &  $n^{\rm (b-jet)}_0$  &  $n^{\rm (b-jet)}_1$  &  $n^{\rm (b-jet)}_2$  & $n^{\rm (b-jet)}_3$ &
    $n^{\rm (b-jet)}_4$ & ${\rm EFF}_{\rm b-tag}$ \\
    \hline
    \hline
    Signal  &  $hh\to 4b$  &   0.1\%    & 3\%     &  25\%     & 53\%     & 20\%      & 8.5\%  \\
    \hline
    \multirow{3}{*}{Background}  &  QCD $4b$  & 1\%      &  8\%    &   27\%   &  44\%     & 20\%       &  8.4\% \\
     &  QCD $2b2j$  &   9\%    & 42\%     &  49\%    & 1\%     &  0.1\%     & 0.04\%  \\
    &  QCD $4j$  &   96\%    &  3.5\%     & 0.5\%     &  0.01\%    & $3\cdot 10^{-4}$\%      &
    $2\cdot 10^{-4}$\%\\
    \hline
  \end{tabular}
  \caption{\small
    The relative fractions  $n^{\rm (b-jet)}_j$ of events for the resolved selection
    for which, out of the four leading small-$R$ jets of the
    event, $j$ jets
    contain at least one $b$-quark with $p_T^b\ge 15$ GeV.
    %
    This information is provided
    for the di-Higgs signal events and for the three QCD background samples.
    %
    The last column indicates the overall predicted
    selection efficiency of the $b$-tagging procedure following
    Eq.~(\ref{btaggingeff})
    \label{tab:btaggingcheck}
  }
  \end{table}
%%%%%%%%%%%%%%%%%%%%%%%%%%%%%%%%%%%%%%%%%%%%%%%

In Table~\ref{tab:btaggingcheck} we collect
the values of $n^{\rm (b-jet)}_j$ for the signal and the three QCD background samples.
%
We find that, rather than the estimate Eq.~(\ref{eq:naive}),
the correct ratio of $b$-tagging selection efficiencies is instead
\be
\frac{{\rm EFF}_{\rm b-tag} \lc 2b2j\rc}{{\rm EFF}_{\rm b-tag} \lc 4b\rc}=
  \frac{0.04\%}{8.4\%} \simeq 5\cdot 10^{-3} \, .
  \ee
  This suppression factor is of the same order of the ratio of $4b$ to $2b2j$ cross-sections
  in the resolved category before $b$-tagging, see Table~\ref{tab:cutflow_noPU_1}.
    %
    This explains why the $2b2j$ contribution cannot be neglected as compared
    to the irreducible $4b$ component of the QCD background.
    %
    A similar calculation from the numbers of Table~\ref{tab:btaggingcheck} shows
    that, on the other hand, the $4j$ component of the background can safely
    be neglected.
    


    %%%%%%%%%%%%%%%%%%%%%%%%%%%%%%%%%%%%%%%%%%%%%%%%%%%%%%%
    %%%%%%%%%%%%%%%%%%%%%%%%%%%%%%%%%%%%%%%%%%%%%%%%%%%%%%%



%
