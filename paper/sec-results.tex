%%%%%%%%%%%%%%%%%%%%%%%%%%%%%%%%%%%%%%%%%%

\section{Results for the cut-based analysis}

\label{sec:results}

In this section we discuss the results of the traditional
cut-based analysis, using the analysis strategy presented in the previous
section.
%
We compare our results with other recent related studies,
and in particular we study how the signal significance
is modified if only the $4b$ QCD background is taken into account,
but the $2b2j$ and $jjjj$ backgrounds are neglected.
%
Then in the next section we show how using multivariate techniques
one can substantially improve the signal significance as compared
to the cut-based analysis.



\subsection{Signal significance}

First of all we compare the cross-sections at various
analysis levels for the three categories, for signal and background,
by providing a detailed cut-flow.
%
Here we consider all backgrounds enumerated in Sect.~\ref{mcgeneration},
and below we discuss how results are modified if only the $4b$
multijet background is used.
%
In particular,
we consider the following levels of the cut-flow
\begin{itemize}
\item {\bf C0}: cross-sections at the generator
  level, before any analysis cuts.
\item {\bf C1}:  cross-sections after the basic jet acceptance selection
  cuts, but before $b$-tagging.
\item {\bf C2}: cross-sections after after the $b$-tagging.
\end{itemize}
Note that {\bf C1} already includes the effect of the Higgs mass
window cut, Eq.~(\ref{higgsmasswindow}).
%
Of course at the {\bf C0} level the numbers will be the same
for the three individual categories.

In the following, results are presented for the exclusive
categorisation, as discussed in Sect.~\ref{sec:categorisation}.
%
In Appendix.~\ref{sec:overlap} we comment
interplay of the overlap between the various categories.
%
At each stage of the cut flow, we also provide the number
of events that would be expected at the HL-LHC with
an integrated luminosity of $\mathcal{L}=3000$ fb$^{-1}$,
as well as 
the signal significance $S/\sqrt{B}$ and the signal
over background ratio $S/B$.
%
We emphasize that it is important not only to achieve a good signal
significance, $S/sqrt{B}$, but also a good signal over background ratio $S/B$:
if the latter is too small, a measurement of the background with an infeasible small
systematic uncertainty would be required.


%%%%%%%%%%%%%%%%%%%%%%%%%%%%%%%%%%%%%%%%%%%%%%%%%%%%%
\begin{table}[t]
  \centering
  \begin{tabular}{c|c|c||c|c||c|c}
    \hline
    \multicolumn{7}{c}{Boosted category}\\
    \hline
    \hline
    &    \multicolumn{2}{c||}{$\sigma$ (fb)}   &  \multicolumn{2}{c||}{$N_{\rm ev}$}
    &   $S/\sqrt{B}$  & $S/B$\\
      &    Signal & Back   &  Signal  & Back
    &   & \\
    \hline
        {\bf C0}  &  36.9  & $9.9\,10^{9}$ & $1.1\,10^5$ & $3.0\,10^{13}$  &  0.020 & $3.7\,10^{-9}$\\
        {\bf C1}  &  0.46    & $1.3\,10^6$    &  $1.4\,10^3$   & $3.9\,10^9$     & 0.022     &  $3.5\,10^{-7}$ \\
        {\bf C2}  &  0.064     &  12.6     &  193   &  $3.8\,10^4$    &  0.99    &  $5.1\,10^{-3}$ \\
        \hline
  \end{tabular}
  $\,$\\
  \vspace{0.4cm}
  \begin{tabular}{c|c|c|c|c|c|c}
    \hline
    \multicolumn{7}{c}{Intermediate category}\\
    \hline
    \hline
    &    \multicolumn{2}{c|}{$\sigma$ (fb)}   &  \multicolumn{2}{c|}{$N_{\rm ev}$}
    &   $S/\sqrt{B}$  & $S/B$\\
      &    Signal & Back   &  Signal  & Back
    &   & \\
    \hline
         {\bf C0}  &  36.9  & $9.9\,10^{9}$ & $1.1\,10^5$ & $3.0\,10^{13}$  &  0.020 & $3.7\,10^{-9}$\\
        {\bf C1}  &   1.44    &   $1.1\,10^{7}$  &  $4.3\,10^3$   &  $3.3\,10^{10}$    &  0.024    &  $1.3\,10^{-7}$ \\
        {\bf C2}  &   0.22    &  722   &  660   &   $2.2\,10^6$   &   0.45   &  $3.1\,10^{-4}$ \\
        \hline
  \end{tabular}
  $\,$\\
  \vspace{0.4cm}
  \noindent
  \begin{tabular}{c|c|c|c|c|c|c}
    \hline
    \multicolumn{7}{c}{Resolved category}\\
    \hline
    \hline
    &    \multicolumn{2}{c|}{$\sigma$ (fb)}   &  \multicolumn{2}{c|}{$N_{\rm ev}$}
    &   $S/\sqrt{B}$  & $S/B$\\
      &    Signal & Back   &  Signal  & Back
    &   & \\
    \hline
       {\bf C0}  &  36.9  & $9.9\,10^{9}$ & $1.1\,10^5$ & $3.0\,10^{13}$  &  0.020 & $3.7\,10^{-9}$\\
        {\bf C1}  &   3.34    & $7.5\,10^{7}$    & $1.1\,10^4$    & $3.3\,10^{10}$     & 0.021     & $4.5\,10^{-8}$  \\
        {\bf C2}  &   0.54    &  $3.5\,10^{3}$   &  $1.6\,10^{3}$   &   $1.1\,10^{7}$   & 0.50     &  $1.5\,10^{-4}$ \\
        \hline
  \end{tabular}
  \caption{\small Cut-flow for the analysis of the boosted (top),
    intermediate (middle) and resolved (bottom)
    categories.
    %
    At each level of the cut-flow, we indicate the cross-sections and the number of
    expected events at the HL-LHC for $\mathcal{L}_{\rm int}=3$ ab$^{-1}$, both for
    signal events and for all the backgrounds combined.
    %
    We also provide in each case the
    signal significance $S/\sqrt{B}$ and the signal
    over background ratio $S/B$.
    %
    The first row {\bf C0} is the generator-level result and thus is common
    to all three categories.
    %
    See text for more details.
    \label{table:cutflow}
  }
\end{table}
%%%%%%%%%%%%%%%%%%%%%%%%%%%%%%%%%%%%%%%%%%%%%%%%%%%%%


The results for the three categories have been collected in
Table~\ref{table:cutflow}.
%
Let us begin with the discussion of the boosted category, that as we have explained
is the one that benefits from a higher signal significance.
%
We see that after all the cuts, including $b$-tagging,
we end up with almost 200 signal events at the HL-LHC, and
still a substantial background of around $40k$ events.
%
The signal significance is around 1.0 at the end of the
cut-based analysis.
%
While it should have been possible to increase this significance
but using more aggressive cuts, we have refrained doing so
since we prefer this optimisation to per performed by
the MVA.


Note that from the results of Table~\ref{table:cutflow}
it is easy to compute the corresponding numbers
for the end of Run II at the LHC with
$\mathcal{L}_{\rm int}=300$ fb$^{-1}$: in the boosted category,
we have only
19 signal events, and the signal significance drops down to
$S/\sqrt{B}\sim 0.3$.
%
We will discuss in the next section what are the Run II prospects
once we include the effects of the MVA in the analysis, but
it is clear already at this level that, unless rates are
increased by new BSM dynamics, one really needs the full
integrated luminosity of the HL-LHC to be able to observe
the Higgs pair production process.

Both the intermediate and resolved categories benefit from higher signal yields,
specially in the resolved category, but this enhancement is canceled by the stronger
increase in the QCD multi-jet background.
%
We see that in both cases the signal significance is around half of that in the boosted category,
with in addition $S/B$ being an order of magnitude smaller.
%
Therefore, the boosted category is clearly determined to be the most useful category,
specially due to the significant suppression of the QCD multi-jet background.
%
And we still have not exploited all the rich information contained on the jet
substructure, as we will do in the next section.

From the results of Table~\ref{table:cutflow}
we see that after the cut-based analysis, the significance of the Higgs pair production
observation in the $4b$ channel using the boosted topology
is $S/\sqrt{B}=0.99$,
and that the combination
of the boosted, intermediate and resolved categories gives a slightly higher
value, around 1.20.
%
We discuss in the next section how this significance can be enhanced by means
of multivariate analysis.

\subsection{Comparison with previous work}

The feasibility of Higgs pair production at the HL-LHC in different
final-state channels
has been explored various groups~\cite{Baur:2003gp,Barger:2013jfa,
  Baur:2003gpa,Barr:2013tda,Dolan:2013rja,
  Dolan:2012rv,Papaefstathiou:2012qe,Gouzevitch:2013qca,Cooper:2013kia,Wardrope:2014kya,deLima:2014dta}, though
the $4b$ final state has received less attention than other final states,
due to the complexity of disentangling the signal over the large
QCD multijet background.
%
Now we compare with two recent studies that have also studied the
feasibility of SM Higgs pair production in the $4b$ final state,
those of the UCL group~\cite{Wardrope:2014kya} and of the
Durham group~\cite{deLima:2014dta}.
%
In order to perform a meaningful comparison, we will only consider
here the $4b$ QCD and $t\bar{t}$ backgrounds, as was done
in these two studies.

First of all let us briefly summarize these two studies.
%
In the UCL group study~\cite{Wardrope:2014kya} is based
on requiring at least four $b$-tagged $R=0.4$ anti-$k_T$ jets
in the central acceptance with $p_T \ge 40$ GeV, which are
then used to form dijets (Higgs candidates) with
$p_T \ge 150$ GeV, $85 \le m_{\rm dijet} \le 140$ GeV
and $\Delta R \le 1.5$ between the two jets that form
each dijet system.
%
In addition to the basic selection cuts, a large number
of additional variables are combined using a
Boosted Decision Tree (BDT) discriminant.
%
This study considers the $b\bar{b}b\bar{b}$ and
$b\bar{b}c\bar{c}$ QCD multijets as well as
$t\bar{t}$, $Zh$, $t\bar{t}h$ and $hb\bar{b}$.
%
Their final significance is found to the $S/\sqrt{B}=2.1$ at the HL-LHC.

The Durham group study~\cite{deLima:2014dta} requires events
to have two $R=1.2$ C/A jets with $p_T\ge 200$ GeV, with
two $b$-tagged subjets inside each large-$R$ jet with
$p_T \ge$ 40 GeV.
%
To improve the signal over background separation, both the BDRS
method and the Shower Deconstruction (SD)~\cite{Soper:2011cr,Soper:2012pb}
technique are applied.
%
The background considered are QCD $4b$ as well as $Zb\bar{b}$, $hZ$ and
$hW$, but no $t\bar{t}$, $2b2j$ or $4j$.
%
At the HL-LHC, the best performance is obtained by requiring two
SD-tagged large-$R$ jets, which leads to $S/\sqrt{B}\sim 2.1$,
with BDRS similar but slightly inferior performance.
%
Therefore, both the Durham and the UCL groups conclude that a signal
significance $S/\sqrt{B}$ slightly above two can  be obtained
in this channel at the HL-LHC.

Since we find that the $2b2j$ and $4j$ backgrounds cannot
be neglected, and previous works did not considered them,
it is interesting to
study how our results are modified if we consider only the QCD $4b$
multi-jet background, but ignore the $2b2j$ and $4j$ backgrounds,
that is, we ignore the contribution from the fakes.
%
These results are shown in
Table~\ref{table:cutflow4B}, which is the analog of
Table~\ref{table:cutflow} but with only the QCD
$4b$ background taken into account.
%
As we can see, these results indicate that given the similarity of the final states
in signal and background events in these cases, the signal significance is
relatively stable at various stages of the cut-flow.
%
As compared to the case in which all backgrounds are taken into account, we
now get a factor two smaller backgrounds: we conclude that in the boosted category
the effect of the fakes is non-negligible, with their contribution being
comparable to that of the irreducible $4b$ QCD multi-jet background.

Concerning the other categories, we see that that ignoring the fakes can lead to a substantial
distortion of the results.
%
In the intermediate category, the signal significance is now much better, with $S/\sqrt{B}$ increasing from
0.5 with all the backgrounds to $1.57$ if only $4b$ are accounted for, a factor 3 improvement.
%
Also the resolved category improves substantially, with signal significance increasing by a factor 2.
%
Therefore, we conclude that a careful modeling of the fakes is very important to truly quantify
the feasibility of Higgs pair production in the $4b$ channel, though for the most
important category, the boosted category, the results are relatively stable.
%
Moreover, we find that the effect of the multijet fake background is of
the same size of the higher-order QCD corrections to $4b$ production~\cite{Binoth:2009rv}.


%%%%%%%%%%%%%%%%%%%%%%%%%%%%%%%%%%%%%%%%%%%%%%%%%%%%%
\begin{table}[t]
  \centering
  \begin{tabular}{c|c|c||c|c||c|c}
    \hline
    \multicolumn{7}{c}{Boosted category}\\
    \hline
    \hline
    &    \multicolumn{2}{c||}{$\sigma$ (fb)}   &  \multicolumn{2}{c||}{$N_{\rm ev}$}
    &   $S/\sqrt{B}$  & $S/B$\\
      &    Signal & Back   &  Signal  & Back
    &   & \\
    \hline
        {\bf C0}  &  36.9  & $1.1\,10^{6}$ & $1.1\,10^5$ & $3.3\,10^{9}$  &  1.90 & $3.3\,10^{-5}$\\
        {\bf C1}  &  0.46    & $1.0\,10^2$    &  $1.4\,10^3$   & $3.1\,10^5$     & 2.48     &  $4.4\,10^{-3}$ \\
        {\bf C2}  &  0.064     &  7.1     &  193   &  $2.1\,10^4$    &  1.30    &  $9.1\,10^{-3}$ \\
        \hline
  \end{tabular}
   $\,$\\
  \vspace{0.4cm}
  \begin{tabular}{c|c|c|c|c|c|c}
    \hline
    \multicolumn{7}{c}{Intermediate category}\\
    \hline
    \hline
    &    \multicolumn{2}{c|}{$\sigma$ (fb)}   &  \multicolumn{2}{c|}{$N_{\rm ev}$}
    &   $S/\sqrt{B}$  & $S/B$\\
      &    Signal & Back   &  Signal  & Back
    &   & \\
    \hline
 {\bf C0}  &  36.9  & $1.1\,10^{6}$ & $1.1\,10^5$ & $3.3\,10^{9}$  &  1.90 & $3.3\,10^{-5}$\\
        {\bf C1}  &  0.46    & $7.2\,10^2$    &  $1.4\,10^3$   & $2.2\,10^6$     & 2.91     &  $1.9\,10^{-3}$ \\
        {\bf C2}  &  0.064     &  58.8     &  193   &  $1.8\,10^5$    &  1.57    &  $3.7\,10^{-3}$ \\
        \hline
  \end{tabular}
  $\,$\\
  \vspace{0.4cm}
  \noindent
  \begin{tabular}{c|c|c|c|c|c|c}
    \hline
    \multicolumn{7}{c}{Resolved category}\\
    \hline
    \hline
    &    \multicolumn{2}{c|}{$\sigma$ (fb)}   &  \multicolumn{2}{c|}{$N_{\rm ev}$}
    &   $S/\sqrt{B}$  & $S/B$\\
      &    Signal & Back   &  Signal  & Back
    &   & \\
    \hline
 {\bf C0}  &  36.9  & $1.1\,10^{6}$ & $1.1\,10^5$ & $3.3\,10^{9}$  &  1.90 & $3.3\,10^{-5}$\\
        {\bf C1}  &  0.46    & $9.0\,10^3$    &  $1.4\,10^3$   & $2.7\,10^7$     & 1.92     &  $3.7\,10^{-4}$ \\
        {\bf C2}  &  0.064     &  986     &  193   &  $2.9\,10^6$    &  0.95  &  $5.5\,10^{-4}$ \\
        \hline
  \end{tabular}
  \caption{\small Same as Table~\ref{table:cutflow}, but now
    only with the QCD $4b$ multi-jet background taken into account.
    %
    \label{table:cutflow4B}
  }
\end{table}
%%%%%%%%%%%%%%%%%%%%%%%%%%%%%%%%%%%%%%%%%%%%%%%%%%%%%


In summary, we find that the results of our cut-based analysis are roughly
consistent with the results of previous related studies.
%
As we show next, once the MVA is used, we manage to outperform these previous
works.


\subsection{Overlap between different categories}
\label{sec:overlap}

As has been discussed in Sect.~\ref{sec:analysis}, while the optimization of each
category is performed in an inclusive analysis, after having determined the order
of significance of the separate categories. we turn to a exclusive analysis,
in order that
we can combine consistently the results from the various topologies without double
counting.
%
Here we discuss the role of the overlap
between  different categories, which is of course ignored in the final exclusive
analyses.
%
The idea is to understand how significant the information is on which events satisfy the requirements
of more than one category at the same time.

To illustrate the role of the overlap between the different categories, in
Fig.~\ref{fig:categorisationHisto} we show the fraction of events that satisfy the requirements
of one or more categories, for signal and background events.
%
We consider the following cases:
\begin{itemize}
\item events that satisfy the requirements of one category (resolved,
  intermediate or boosted),
\item events that satisfy the conditions of both the resolved and the intermediate categories,
\item and the same for events that fall both
  into the resolved and boosted categories.
\end{itemize}
 Note that by construction, there is no overlap between the boosted
  and intermediate categories due to the
 orthogonal jet multiplicity cuts.

%%%%%%%%%%%%%%%%%%%%%%%%%%%%
\begin{figure}[t]
\begin{center}
\includegraphics[width=0.65\textwidth]{plots/overlap_categories_C1.pdf}
\caption{\small The fraction of events that satisfy the requirements
  of one or more categories, for signal and background events.
  %
  By construction, there is no overlap between the boosted
  and intermediate categories due to the
   orthogonal jet multiplicity cuts.
}
\label{fig:categorisationHisto}
\end{center}
\end{figure}
%%%%%%%%%%%%%%%%%%%%%%%

As we can see from Fig.~\ref{fig:categorisationHisto}, an large majority of the events
belong to the resolved category, specially in the background case ($\sim 90\%$) but
also a large part of the signal ($\sim 60\%$).
%
In all the other categories, including
the overlaps, signal events are more likely to belong to them than
background events.
%
In the intermediate category, we have around $\sim$30\% of the signal
events and $\sim$10\% of the background events.
%
For the boosted case, we find $\sim$ 6\% of signal events and
$\sim$2\% of background events.
%
We also find that the overlap between the boosted and the other categories
is small: none with the intermediate (by construction), and then $\sim$3\%
($\sim$0.05\%) of the signal (background) events.
%
The overlap between the resolved and the intermediate is also smaller
than the two categories separately: $\sim$7\%
($\sim$ 3\%) for signal (background) events.

Therefore, from the results of Fig.~\ref{fig:categorisationHisto}
we can conclude that the effect of the overlap between categories would
be small correction with respect to the exclusive selection approach
that we use in this work.
%
Note also that while the majority of signal events are in the resolved
category, since the fraction of background events is also larger,
it is advantageous to use the small fraction of events in the boosted category
since the suppression of the background in this category is larger
than in the others.

\subsection{Background decomposition}

Up to now we have considered the effects of all backgrounds added
together, except in Table~\ref{table:cutflow4B} and the corresponding
discussion where we have studies how our
results are modified if only the QCD $4b$ background is included.
%
In this section we study the decomposition of the background in each
of its individual components, for the three different categories
of events used in this work.
%

In Table~\ref{table:cutflowBack}   we show the results for the cross-sections for the different
backgrounds that we consider at the end of the cut-based analysis, for each
of the three categories.
%
All the backgrounds have been normalized to known higher-order results,
as summarized in Table~\ref{tab:samples}.
%
As we can see from this comparison, as opposed that what naively one would expect,
the QCD $2b2j$ background turns out not to be negligible as compared to the $4b$
multijets in any of the categories.

%%%%%%%%%%%%%%%%%%%%%%%%%%%%%%%%%%%%%%%%%%%%%%%%%%%%%
\begin{table}[t]
  \centering
  \begin{tabular}{c|c|c|c}
    \hline
    Background  &    \multicolumn{3}{c}{$\sigma$ (fb)}   \\
    \hline
  &    Resolved  &  Intermediate  &   Boosted \\
    \hline
    \hline
  QCD $4b$   &   $9.9\,10^{2}$       &  $59$       &   $7.1$        \\
  QCD $2b2j$ &   $2.4\,10^{3}$       &  $6.4\,10^{2}$       &   $3.5$        \\
  QCD $4j$ &     $1.7\,10^{2}$       &   $22$      &    $2.1$        \\
  $t\bar{t}$ &   $3.6$       &  $2.2$       &   $4.5\,10^{-2}$        \\
  \hline
  \hline
Total     &    $3.6\,10^{3}$      &   $7.2\,10^{2}$      &     $13$      \\
\hline
\end{tabular}
    \caption{\small The cross-sections for each of the individual background
      components (and their total sum) at the end of the cut-based
      analysis, for each of three categories.
    \label{table:cutflowBack}
    }
\end{table}
%%%%%%%%%%%%%%%%%%%%%%%%%%%%%%%%%%%%%%%%%%%%%%%%%%%%%

In the boosted case, the sum of the $2b2j$ and $4j$ components is almost as large as the
irreducible $4b$ background, thus increasing the total background rates by a factor 2 as compared
to the case where only $4b$ is accounted for.
%
This effect is more dramatic in the intermediate and resolved categories: in the former,
$2b2j$ completely dominates the background (thus the increase in $S/sqrt{B}$ noted in
Table~\ref{table:cutflow4B}  when only the $4b$ component is kept), and in the latter
it is comparable to $4b$.
%
Therefore, we conclude that an accurate estimate of the fake $b$-tag contribution is
an essential component of this analyses, and that complete QCD multijet samples
should be included.


  {\bf add plots BKG}

  One might ask why the $2b2j$ process
  is not subdominant as compared to $4b$: while its generator-level cross-section is
  much larger, $\sigma_{2b2j} \simeq 240\sigma_{4b}$, one expects a suppression
  of the order of $f_l^2 =10^{-4}$ since only with two light jets mistagged as $b$-jets the event
  would be identified as a Higgs candidate.
  %
  We have checked that this expectation is true only at parton level: once parton shower effects
  are accounted for, both the radiation of $b\bar{b}$ pairs from the shower and combinatorics increase
  the number of $b$ quarks in the final state, enhancing substantially the $2b2j$ rates as compared
  to the naive parton level expectation.
  %
  To verify this, we plot the number of $b$-quarks.....
