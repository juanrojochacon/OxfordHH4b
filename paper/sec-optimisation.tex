
\section{Optimisation}
\label{sec:optimisation}

One important motivation of this work was to quantify which aspects
of the performance of the LHC detectors needs to be improved
the most in order to maximise the significance of the observation
of double Higgs production in the $4b$ final state.
%
In this section we explore how the results with the baseline settings,
summarized in the previous section, are modified when some of
these settings are changed.
%
In particular we will explore the dependence of the results on
the $b$-tagging probability $f_b$, the light jet mistag rate
$f_l$, as well on the momentum smearing fraction.
%
The variations of the analysis
settings that we will explore are collected in
Table~\ref{sec:variations}.

%%%%%%%%%%%%%%%%%%%
\begin{table}[h]
  \centering
  \begin{tabular}{|c|c|c|c|}
\hline
    Scenario  &  $f_b$  &  $f_l$  &  $\Delta p_T$ \\
    \hline
    \hline
    Baseline  &  0.8   &   0.01  &  5\% \\
    \hline
    A        &  0.9   &   0.01  &  5\% \\
    B        &  0.7   &   0.01  &  5\% \\
    C        &  0.8   &   0.005  &  5\% \\
    D        &  0.8   &   0.02  &  5\% \\
    E        &  0.9   &   0.005  &  2\% \\   
    \hline
  \end{tabular}
  \caption{\small The baseline settings for the $b$-jet
    tagging probability, the light jet mistag rate $f_l$
    and the $p_T$ resolution $\Delta p_T$, compared
    to the various scenarios that we discussed in this section.
\label{sec:variations}
  }
  \end{table}
%%%%%%%%%%%%%%%%%%%

We also study how the signal significance is modified as a function
of the systematic uncertainty that needs to be obtained
in the measurement of the background processes.
%
This is specially important in a process like this one where
the signal over background ratio $S/B$ is small, and thus were
it is essential to measure the backgrounds as accurately
as possible.


Given that we have found in the previous section that the
only category with real discrimination power is the boosted one,
in this section we concentrate only on this one,
by comparing our baseline results from Table~\ref{table:cutflowMVA}
with the new scenarios.
%
When necessary, the MVA has been retrained to optimise the
new information contained in the various scenarios.

The results of this optimisation study are collected in
Table~\ref{table:cutflowMVAoptimisation}

%%%%%%%%%%%%%%%%%%%%%%%%%%%%%%%%%%%%%%%%%%%%%%%%%%%%%
\begin{table}[t]
  \centering
  \begin{tabular}{c||c|c|c|c}
    \hline
    \multicolumn{5}{c}{Boosted category}\\
    \hline
    \hline
 Scenario &    \multicolumn{2}{c|}{$N_{\rm ev}$} &  $S/\sqrt{B}$  & $S/B$ \\
       &   Signal & Back   &     &    \\
 \hline
 \hline
   Baseline   & 107 & 1040 & 3.31  & 0.10\\
   \hline
   &  &   &   &   \\
   &  &   &   &   \\
   &  &   &   &   \\
   &  &   &   &   \\
         &  &   &   &   \\
   \hline
  \end{tabular}
  \caption{\small The details of the various scenarios have been collected in
    Table~\ref{sec:variations}.
    \label{table:cutflowMVAoptimisation}
  }
\end{table}
%%%%%%%%%%%%%%%%%%%%%%%%%%%%%%%%%%%%%%%%%%%%%%%%%%%%%
