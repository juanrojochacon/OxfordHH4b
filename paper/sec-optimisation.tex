
\section{Analysis optimisation}
\label{sec:optimisation}

One motivation for this work was to quantify which aspects
of the performance of the LHC detectors need to be improved
the most in order to maximise the significance of the observation
of double Higgs production in the $4b$ final state.
%
In this section we explore how the post-MVA
results with the baseline settings,
summarized in the previous section, are modified when some of
these settings are modified.
%
In particular we will explore the dependence of the results on
the $b$-tagging probability $f_b$, the light jet mistag rate
$f_l$, as well on the four-momentum smearing fraction $\sigma_E$,
see Eq.~(\ref{eq:smearing}).
%
The variations of the analysis
settings that we will explore are collected in
Table~\ref{sec:variations}.

%%%%%%%%%%%%%%%%%%%
\begin{table}[h]
  \centering
  \begin{tabular}{|c|c|c|c|}
\hline
    Scenario  &  $f_b$  &  $f_l$  &  $\sigma_E$ \\
    \hline
    \hline
    Baseline  &  0.8   &   0.01  &  5\% \\
    \hline
    A        &  0.9   &   0.01  &  5\% \\
    B        &  0.7   &   0.01  &  5\% \\
    C        &  0.8   &   0.005  &  5\% \\
    D        &  0.8   &   0.02  &  5\% \\
    E        &  0.9   &   0.005  &  3\% \\
    F        &  0.9   &   0.005  &  7\% \\   
    \hline
  \end{tabular}
  \caption{\small The baseline settings for the $b$-jet
    tagging probability, the light jet mistag rate $f_l$
    and the momentum resolution $\sigma_E$, compared
    to the various scenarios that we discussed in this section.
\label{sec:variations}
  }
  \end{table}
%%%%%%%%%%%%%%%%%%%


Given that we have found in the previous section that the
category with highest discrimination power is the boosted one,
in this section we concentrate on it,
by comparing our baseline results from Table~\ref{table:cutflowMVA}
with the new scenarios.
%
Each time of the parameters in Table~\ref{sec:variations}
 has been changed, the MVA has been retrained to optimise the
new information contained in the various scenarios.
%
This is important since for example, when we vary $f_b$
or $f_l$, the relative composition of the multijet
background will change, and this can translate into
modifications of the kinematical variables used for
the MVA training.
%
Similarly, if the momentum resolution is modified,
the weight of the different variables in the MVA
will be altered, in particular the invariant mass
of Higgs candidates $m_h$.


The results of this optimisation study are collected in
Table~\ref{table:cutflowMVAoptimisation}.
%
There we compare the results for the number of signal
and background events $N_{\rm ev}$ expected at the
HL-LHC, the signal significance, and the $S/B$ ratio,
for our nominal settings and for the variations
of the analysis settings.

%%%%%%%%%%%%%%%%%%%%%%%%%%%%%%%%%%%%%%%%%%%%%%%%%%%%%
\begin{table}[t]
  \centering
  \begin{tabular}{c||c|c|c|c}
    \hline
    \multicolumn{5}{c}{Boosted category}\\
    \hline
    \hline
 Scenario &    \multicolumn{2}{c|}{$N_{\rm ev}$} &  $S/\sqrt{B}$  & $S/B$ \\
       &   Signal & Back   &     &    \\
 \hline
 \hline
   Baseline   & 107 & 1040 & 3.31  & 0.10\\
   \hline
   &  &   &   &   \\
   &  &   &   &   \\
   &  &   &   &   \\
   &  &   &   &   \\
         &  &   &   &   \\
   \hline
  \end{tabular}
  \caption{\small
Number of signal
and background events $N_{\rm ev}$ expected at the
HL-LHC, signal significance and $S/B$ ratio
for the nominal settings and for the variations
of the analysis settings, summarized
in  Table~\ref{sec:variations}.
 \label{table:cutflowMVAoptimisation}
  }
\end{table}
%%%%%%%%%%%%%%%%%%%%%%%%%%%%%%%%%%%%%%%%%%%%%%%%%%%%%

Neglecting the
contribution from light jet fakes,
it is easy to estimate how $S/\sqrt{B}$ should
scale with $f_b$: since we have two double-$b$-tagged jets
both in signal and background events, one should find that
\be
\lp \frac{S}{\sqrt{B}} \rp _{f_b}\Bigg/
\lp \frac{S}{\sqrt{B}} \rp_{f_b'}
\simeq \frac{f_b^2}{f_b'^2} \, .
\ee
By comparing this expectations with the numbers of
Table~\ref{table:cutflowMVAoptimisation}, we see that there
is a reasonable agreement in the boosted category,
as expected since it is dominated by the irreducible
$4b$ component of the QCD multijet background.

The most aggressive scenario is the one in which the $b$-tagging
is improved up an efficiency $f_b=0.9$, while at the same
time the light jet fake rate is reduced down to $0.5\%$, while
at the same time the momentum resolution has been decreased
to $\sigma_E=3\%$.
%
In this scenario, the signal significance of the boosted category
increases to AAA, to be compared to BBB with the baseline settings.
